\documentclass{article}

\usepackage{mathtools,amsfonts}
\usepackage{enumerate}

\begin{document}
\thispagestyle{empty}

\begin{center}
  \textbf{\Large Intermediate Test 1}
  % LEVEL is Senior, Intermediate or Beginner
  % NUMBER is the test number: 1, 2, etc.
  \\ \vspace{1em}
  \textbf{\large Stellenbosch Camp 2019}
  \\ \vspace{1em}
  \textbf{\large Time: $2\frac{1}{2}$ hours}
\end{center}


\begin{enumerate}[1.]

\item % Standard
If $x + \frac{1}{x} = 3$, what is the value of $x^5 + \frac{1}{x^5}$?\\



\item % Source
Given a triangle $ABC$ and two points $M$ and $N$ on sides $AB$ and $AC$ respectively. Let $BN$ and $CM$ intersect at $P$. It is given that the areas of $\triangle CPN$, $\triangle BPM$ and $\triangle BPC$ are $4$, $6$ and $5$ respectively.
Find the area of $\triangle ABC$.\\
(Bonus: if you would like an extra mark, use the areas 20, 19, 2019 instead)\\



\item % Source
Find all positive integers $n$ where the product of the positive factors of $n$ is $n^3$.\\



\item % Source
A set $T$ of integers is called \textit{broken} if there are integers $a < b < c$ such that $a$ and $c$ are in $T$, but $b$ is not in $T$.\\
Find the number of broken subsets of $\{1, 2, \dots, 2019\}$.\\



\item % Venezuela Final Round 2019, Q4
Let $ABC$ denote an equilateral triangle. Let $M$ and $N$ denote the midpoints of $AB$ and $BC$, respectively.
Let $P$ be a point outside $ABC$ such that $APC$ is isosceles and right-angled at $P$. Lines $PM$ and $AN$ meet at $I$.
Prove that $CI$ is the angle bisector of $\angle ACM$.\\

\end{enumerate}


\vfill
% ASCII art
\begin{center} \begin{verbatim}
       .___,   
    ___('v')___
    `"-\._./-"'
        ^ ^ 

\end{verbatim} \end{center}

\end{document}
