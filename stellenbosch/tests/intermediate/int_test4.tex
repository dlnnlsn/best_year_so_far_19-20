\documentclass{article}

\usepackage{mathtools,amssymb}
\usepackage{enumerate}
\usepackage{fancyvrb}
\usepackage{fullpage}


\title{Intermediate Test 4}
\author{Stellenbosch Camp 2019}
\date{Time: $2\frac{1}{2}$ hours}

\begin{document}
\maketitle
\thispagestyle{empty}

\hfill\textit{Each question is worth 7 marks.}

\vfill
\vfill


\begin{enumerate}[1.]

\item % Talent Search 2004 year book
Prove that for any natural number $n$, $n^5-5n^3+4n$ is divisible by $120$. 


\vfill

\item % Talent Search 1994 year book
Four pair of socks are hung out side by side on a straight washing line.
The socks in each pair are identical but the pair themselves are different colours.
How many different colour patterns can be made if no sock is allowed to be next to its matching pair?


\vfill

\item % Source The Emile
Let $n\geq3$ be a positive integer.
Determine all $n$ such that it is possible to find a pair of diagonals of a regular $n$-gon which intersect at $90^\circ$.


\vfill

\item % Source Romania Junior 2018 Q1
\newcommand{\floorsqrt}[1]{\left\lfloor\sqrt{#1}\right\rfloor}
Find the positive integer solutions to the equation
\[ \floorsqrt{8n+1} +\floorsqrt{8n+2} +\dotsb +\floorsqrt{8n+7} = 2027. \]


\vfill

\item % Source Irish 2018 Q5
Let $x_0, x_1,..., x_n$ be real numbers and define
\[y_k=x_k-x_{n-k}, \quad k=0,1,...,n.\]
Prove that 
\[y_0^2+ y_1^2+...+ y_n^2 \leq 4(x_0^2 + x_1^2 + ... + x_n^2) \]
and determine when equality holds.


\end{enumerate}


\vfill
\vfill
% ASCII art
\begin{center}
\begin{BVerbatim}



                          ___
                       .-"-~-"-.
                      /.-"-.-"-.\
                      ||((o|o))||
                      )\__/V\__/(
                     / ~ -...- ~ \
                    |\` ~. ~ .~ `/|
                 () | `~ - ^ - ~` |
             () //  | ;  '  :  .  |
            ()\\/_() \ . : '  ; '/
           ___/ /_____'.   ; ' .'____
                 _   ^ `uu---uu`    /\
          __jgs________^ _________^_\/
                       \ \
                       //\\()
                     ()/  ()
                        ()

\end{BVerbatim}
\end{center}

\end{document}
