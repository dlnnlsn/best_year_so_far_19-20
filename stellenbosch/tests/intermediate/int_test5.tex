\documentclass{article}

\usepackage{mathtools,amssymb}
\usepackage{enumerate}
\usepackage{fancyvrb}
\usepackage{fullpage}


\title{Intermediate Test 5}
\author{Stellenbosch Camp 2019}
\date{Time: $4$ hours}

\begin{document}
\maketitle
\thispagestyle{empty}

\hfill\textit{Each question is worth 7 marks.}

\vfill
\vfill


\begin{enumerate}[1.]

\item % 


\vfill

\item % 


\vfill

\item % 


\vfill

\item % 


\vfill

\item % 


\vfill

\item % Austrian MO Q15, C
In the country of Oddland, there are stamps with values $1$ cent, $3$ cents, $5$ cents, etc., one type for each odd number.
the rules of Oddland Postal Services stipulate the following: for any two distinct values, the number of stamps of the higher value on an envelope must never exceed the number of stamps of the lower value.

In the country of Squareland, on the other hand, there are stamps with values $1$ cent, $4$ cents, $9$ cents, etc., one type for each square number.
Stamps can be combined in all possible ways in Squareland without additional rules.

Prove that for every positive integer $n$:
In Oddland and Squareland there are equally many ways to correctly place stamps of a total value $n$ cents on an envelope.
Rearranging the stamps on an envelope makes no difference.


\end{enumerate}


\vfill
\vfill

% ASCII art
\begin{center}
\begin{BVerbatim}
% insert ASCII art here
\end{BVerbatim}
\end{center}

\end{document}
