\documentclass{article}

\usepackage{mathtools,amsfonts}
\usepackage{enumerate}

\begin{document}

\begin{center}
  \textbf{\Large Intermediate Test 1 Solutions}
  % LEVEL is Advanced, Intermediate or Beginner
  % NUMBER is the test number: 1, 2, etc.
  \\ \vspace{1em}
  \textbf{\large Stellenbosch Camp 2019}
\end{center}


\begin{enumerate}[1.]

\item[1.] %Standard
\textit{If $x + \frac{1}{x} = 3$, what is the value of $x^5 + \frac{1}{x^5}$?}

\textit{Solution}: 
$$3^2 = (x + \frac{1}{x})^2 = x^2 + 2x(\frac{1}{x}) + \frac{1}{x^2} = x^2 + 2 + \frac{1}{x^2}$$
$$\Rightarrow x^2 + \frac{1}{x^2} = 3^2 - 2 = 7$$
$$7 \times 3 = (x^2 + \frac{1}{x^2})(x + \frac{1}{x}) = x^3 + x + \frac{1}{x} + \frac{1}{x^3} = x^3 + \frac{1}{x^3} + 3$$
$$\Rightarrow x^3 + \frac{1}{x^3} = 21 - 3 = 18$$
$$7^2 = (x^2 + \frac{1}{x^2})^2 = x^4 + 2x^2(\frac{1}{x^2}) + \frac{1}{x^4} = x^4 + 2 + \frac{1}{x^4}$$
$$\Rightarrow x^4 + \frac{1}{x^4} = 7^2 - 2 = 47$$
$$\Rightarrow 47 \times 3 = (x^4 + \frac{1}{x^4})(x + \frac{1}{x}) = x^5 + x^3 + \frac{1}{x^3} + \frac{1}{x^5} = x^5 + \frac{1}{x^5} + 18$$
$$\Rightarrow x^5 + \frac{1}{x^5} = 141 - 18 = 123$$

\vspace{6.81mm}

\item[2.] %Standard
\textit{Given a triangle $ABC$ and two points $M$ and $N$ on sides $AB$ and $AC$ respectively. Let $BN$ and $CM$ intersect at $P$. It is given that the areas of $\triangle CPN$, $\triangle BPM$ and $\triangle BPC$ are $4$, $6$ and $5$ respectively.
Find the area of $\triangle ABC$.\\
(Bonus: if you would like an extra mark, use the areas 20, 19, 2019 instead)}

\textit{Solution}: Problem solution

\vspace{6.81mm}

\item[3.] %Switzerland Preliminary 2011
\textit{Find all positive integers $n$ where the product of the positive factors of $n$ is $n^3$.}

\textit{Solution}: Notice that $n = 1$ is a trivial solution. Suppose that $p$ is a factor of $n$, then $\frac{n}{p}$ is an integer and it is also a factor of $n$. If $p \le \sqrt{n}$, then $\frac{n}{p} \ge \frac{n}{\sqrt{n}} = \sqrt{n}$.
This means that factors come in unique pairs on either side of $\sqrt{n}$ with the product of these factors being $n$. If $n$ is a perfect square, then there is some $k$ such that $k^2 = n$. This factor will be paired with itself, so it must be dealt with separately.
If the product of the factors of $n$ is $n^3$, then the product of factors excluding $k$ must be $n^{2.5}$. Since the product of pairs of factors is always $n$, the product of all pairs must be an integer power of $n$. Thus, it is not possible to get a product of factors being $n^{2.5}$. This shows that $n$ cannot be a perfect square.\\

Furthermore, since the product of factors in a pair is always $n$, the total number of pairs must be $3$ so that the product of all factors is $n^3$. This means that $n$ must have $6$ factors in total.\\

If $n = p_1^{q_1} \times p_2^{q_2} \times \dots \times p_r^{q_r}$ where all $p_i$ are unique primes and all $q_i$ are positive integers, then the number of factors of $n$ is $(q_1 + 1) \times (q_2 + 1) \times \dots \times (q_r + 1)$. Since the factors of $6$ are $1$, $2$, $3$ and $6$, the possible combinations are:
$$q_1 + 1 = 2, q_2 + 1 = 3 \Rightarrow q_1 = 1, q_2 = 2$$
$$q_1 + 1 = 6 \Rightarrow q_1 = 5$$

This shows that $n$ must either be represented as $n = p^1 \times q^2$ or $n = p^5$ where $p$ and $q$ are unique primes. A simple check of the product of factors shows all numbers of this form work.
$$n = p^1 \times q^2 \Rightarrow 1 \times p \times q \times pq \times q^2 \times pq^2 = p^3q^6 = n^3$$
$$n = p^5 \Rightarrow 1 \times p \times p^2 \times p^3 \times p^4 \times p^5 = p^{15} = n^3$$

Therefore, the product of factors of $n$ is $n^3$ if and only if $n = 1$, $n = p \times q^2$ or $n = p^5$ where $p$ and $q$ are unique prime numbers.

\vspace{6.81mm}

\item[4.] %Spanish MO Q1
\textit{A set $T$ of integers is called \textit{broken} if there are integers $a < b < c$ such that $a$ and $c$ are in $T$, but $b$ is not in $T$.\\
Find the number of broken subsets of $\{1, 2, \dots, 2019\}$.}

\textit{Solution}: The opposite of a broken set is a contiguous set or the empty set. To count the number of broken subsets, we will find the total number of subsets and subtract the number of contiguous subsets.
The total number of subsets of $\{1, 2, \dots, 2019\}$ is $2^{2019}$ since each element is either in a subset or not in it. There is one subset of size $0$ and the number of contiguous subsets of size $s$ is $2019 - s + 1$.\\
Therefore, the number of broken sets is $2^{2019} - 1 - \sum_{s=1}^{2019}(2019 - s + 1) = 2^{2019} - 1 - \sum_{s=1}^{2019}s = 2^{2019} - 1 - \frac{2019 \times 2020}{2}$.

\vspace{6.81mm}

\item[5.] %Venezuela Final Round Q4
\textit{Let $ABC$ denote an equilateral triangle. Let $M$ and $N$ denote the midpoints of $AB$ and $BC$, respectively.
Let $P$ be a point outside $ABC$ such that $APC$ is isosceles and right-angled at $P$. Lines $PM$ and $AN$ meet at $I$.
Prove that $CI$ is the angle bisector of $\angle ACM$.}

\textit{Solution}: Since $\triangle BAN \equiv \triangle CAN$ we have $\angle IAB = \angle NAB = \angle CAN = \angle CAI $, and thus $AI$ is an angle bisector of $\angle CAM$. Note that $\triangle AMC \equiv \triangle BMC$ we have $\angle AMC = \angle BMC$ and $\angle AMC + \angle BMC =180^\circ$, thus $\angle AMC = \angle BMC = 90^\circ$. Since $\angle AMC + \angle CPA = 90^\circ +90^\circ =180^\circ$ we have that AMCP is a cyclic quadrilateral. So $\angle AMI = \angle AMP = \angle ACP = \angle CAP = \angle CMP = \angle CMI$. Therefore $MI$ is a bisector of $\angle AMC$. This implies that $I$ is the incentre of $\triangle AMC$ and so $CI$ is the angle bisector of $\angle ACM$ by necessity.


\end{enumerate}

\end{document}
