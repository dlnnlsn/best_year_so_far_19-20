\documentclass{article}

\usepackage{mathtools,amsfonts, amssymb}
\usepackage{enumerate}
\usepackage{amsmath}
\usepackage{fullpage}

\begin{document}

\begin{center}
  \textbf{\Large Intermediate Test 4 Solutions}
  \\ \vspace{1em}
  \textbf{\large Stellenbosch Camp 2019}
\end{center}

\vspace{12pt}

\begin{enumerate}[1.]

\item[1.] % Talent Search 2004 year book
\textit{Prove that for any natural number $n$, $n^5-5n^3+4n$ is divisible by $120$. }

\textit{Solution}:
\begin{align*}
  n^5 - 5n^3 + 4n &= n(n^4 - 5n^2 + 4) \\
  &= n(n^2 - 1)(n^2 - 4) \\
  &= n(n - 1)(n + 1)(n - 2)(n + 2) \\
  &= (n - 2)(n - 1)n(n + 1)(n + 2)
\end{align*}
Therefore, the original expression can be written as the product of $5$ consecutive integers. Notice that amongst any $5$ consecutive integers, one of them must be a multiple of $5$, at least one must be a multiple of $4$, at least one must be a multiple of $3$ and there must be at least one that is divisible by $2$, but not $4$.
Therefore, the product of any $5$ consecutive integers is divisible by $5 \times 4 \times 3 \times 2 = 120$. This means that $n^5 - 5n^3 + 4n$ is divisible by $120$.

\item[2.] % Talent Search 1994 year book
\textit{Four pair of socks are hung out side by side on a straight washing line.
The socks in each pair are identical but the pair themselves are different colours.
How many different colour patterns can be made if no sock is allowed to be next to its matching pair?
}

\textit{Solution}:
Call the socks a, a, b, b, c, c, d, d.
There are $4 \times 3 \times 2$ ways of selecting the first three socks as a, b, c (all different).\\
$$
\left.
\begin{matrix}
abc & a & bdcd\\
& & cdbd\\
&& dbdc \\
&& dbcd  \\
&&dcdb \\
&&dcbd
\end{matrix} \right\} 6
$$
$$
\left.
\begin{matrix}
abc & b & adcd\\
& & cdad\\
&& dadc \\
&& dacd  \\
&&dcda \\
&&dcad
\end{matrix} \right\} 6
$$
$$
\left.
\begin{matrix}
abc & da & bcd\\
& & bdc\\
&& cbd \\
&& cdb  \\
&&dbc \\
&&dcb
\end{matrix} \right\} 6
$$
$$
\left.
\begin{matrix}
abc & db & acd\\
& & adc\\
&& cda \\
&& cad  \\
&&dac \\
&&dca
\end{matrix} \right\} 6
$$
$$
\left.
\begin{matrix}
abc & dc & abd\\
& & adb\\
&& dba \\
&& dab  \\
&&bda \\
&&bad
\end{matrix} \right\} 6
$$
and so there are a total of $(4 \times 3 \times 2) \times (6 \times 5) = 720$ ways here.\\
There are $4 \times 3$ ways of selecting socks a, b, a (third same as first).
$$
\left.
\begin{matrix}
aba & b & cdcd\\
& & dcdc
\end{matrix} \right\} 6
$$
$$
\left.
\begin{matrix}
aba & c & bdcd\\
& & dbcd\\
&& dbdc \\
&& dcbd  \\
&&dcdb
\end{matrix} \right\} 6
$$
$$
\left.
\begin{matrix}
aba & d & bcdc\\
& & cbdc\\
&& cbcd \\
&& cdbc  \\
&&cdcb
\end{matrix} \right\} 6
$$
and so there are $4 \times 3 \times (2 + 5 + 5) = 144$ ways here.\\
So the total number of arrangements is $720 + 144 = 864$.


\item[3.] % Source The Emile
\textit{Let $n\geq3$ be a positive integer.
Determine all $n$ such that it is possible to find a pair of diagonals of a regular $n$-gon which intersect at $90^\circ$.}

\textit{Solution}:
Clearly, $n = 3$ is not possible. Let circle $\Gamma$, with center $O$, circumscribe the $n$-gon, and let $A$, $B$, $C$, and $D$ be some of its vertices.

Note that $\angle COB = \frac{360^o}{n} \times n_{BC}$, where $n_{BC}$ is the number of sides between $B$ and $C$.
$$\implies \angle CAB = \frac{180^o}{n} \times n_{BC}$$

For any two diagonals of the $n$-gon, $AB$ and $CD$ to intersect at $90^o$, we must have that:
\begin{align*}
  90^o = \angle APC &= \angle DAB + \angle ADC \\
  &= n_{BD} \times \frac{180^o}{n} + n_{AC} \times \frac{180^o}{n} \\
  &= \frac{180^o}{n} (n_{BD} + n_{AC}) \\
  \implies n &= \frac{180^o}{90^o} (n_{BD} + n_{AC}) \\
  &= 2 (n_{BD} + n_{AC}) \\
  \implies 2 &\; |\; n
\end{align*}

A construction for all even $n$ is to choose $AC$ such that $A$ and $C$ are directly opposite on the polygon and to choose $B$ and $D$ such that $D$ is the reflection of $B$ over $AC$. This shows that all even $n$ can be constructed and no odd $n$ can be constructed.


\item[4.] % Source Romania Junior 2018 Q1
\newcommand{\floorsqrt}[1]{\left\lfloor\sqrt{#1}\right\rfloor} 
\textit{
Find the positive integer solutions to the equation
\[ \floorsqrt{8n+1} +\floorsqrt{8n+2} +\dotsb +\floorsqrt{8n+7} = 2027. \]}

\textit{Solution}:
Notice that for $n > 1$, $\floorsqrt{8n + 1}$ and $\floorsqrt{8n + 7}$ differ by at most $1$. This means that all values $\floorsqrt{8n + 1}$, $\floorsqrt{8n + 2}$, $\dots$, $\floorsqrt{8n + 7}$ are either the same or differ by 1. Suppose that there are $a$ values that differ by $1$ and let $\floorsqrt{8n + 1} = k$.
$$7k + a = 2027$$
Notice that $2027 \equiv _7 4 \implies a = 4 \implies k = 289$ From this, we see that $\floorsqrt{8n + 1} = 289$ and specifically $\sqrt{8n + 4} = 290 \implies 8n + 4 = 290^2$. Therefore, the only value that works is $n = \frac{290^2 - 4}{8} = \frac{145^2 - 1}{2}$ which is an integer.

\item[5.] % Source Irish 2018 Q5
\textit{
Let $x_0, x_1,..., x_n$ be real numbers and define
\[y_k=x_k-x_{n-k}, \quad k=0,1,...,n.\]
Prove that 
\[y_0^2 + y_1^2+...+ y_n^2 \leq 4(x_0^2 + x_1^2 + ... + x_n^2) \]
and determine when equality holds.
}

\textit{Solution}: 
Substitute the definition of all $y_i$ into the inequality:
\begin{align*}
  &&\sum_{i = 0}^{n}y_i^2 &\leq 4\sum_{i = 0}^{n}x_i^2 &\\
  &\iff& \sum_{i = 0}^{n}(x_i - x_{n - i})^2 &\leq 4\sum_{i = 0}^{n}x_i^2 &\\
  &\iff& 2\sum_{i = 0}^{n}x_i^2 - 2\sum_{i = 0}^{n}x_ix_{n - i} &\leq 4\sum_{i = 0}^{n}x_i^2& \\
  &\iff& 0 &\leq 2\sum_{i = 0}^{n}x_i^2 + 2\sum_{i = 0}^{n}x_ix_{n - i} \\
  &\iff& 0 &\leq \sum_{i = 0}^{n}(x_i + x_{n - i})^2 &
\end{align*}

Since the square of a real number is always non-negative, the sum on the right must also be non-negative. Equality only holds when
$$x_i + x_{n - i} = 0$$
for all choices of $i$ with $0 \le i \le n$.

\end{enumerate}

\end{document}
