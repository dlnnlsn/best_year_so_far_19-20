\documentclass{article}

\usepackage{mathtools,amsfonts}
\usepackage{enumerate}
\usepackage{amsmath}
\usepackage{fullpage}

\begin{document}

\begin{center}
  \textbf{\Large Intermediate Test 1 Solutions}
  % LEVEL is Advanced, Intermediate or Beginner
  % NUMBER is the test number: 1, 2, etc.
  \\ \vspace{1em}
  \textbf{\large Stellenbosch Camp 2019}
\end{center}


\begin{enumerate}[1.]

\item[1.] % Source
\textit{}

\textit{Solution}:
\vspace{6.81mm}

\item[2.] % Standard
\textit{Find all positive integers $n$ such that $\frac{n^2 + 8n + 51}{n + 4}$ is also a positive integer.}

\textit{Solution}:
$$\frac{n^2 + 8n + 51}{n + 4} = n + \frac{n^2 + 8n + 51 - n(n + 4)}{n + 4}$$
$$\Rightarrow \frac{n^2 + 8n + 51}{n + 4} = n + \frac{4n + 51}{n + 4}$$
$$\Rightarrow \frac{n^2 + 8n + 51}{n + 4} = n + 4 + \frac{4n + 51 - 4(n + 4)}{n + 4}$$
$$\Rightarrow \frac{n^2 + 8n + 51}{n + 4} = n + 4 + \frac{35}{n + 4}$$

This shows that if $\frac{n^2 + 8n + 51}{n + 4}$ is a positive integer, then $\frac{35}{n + 4}$ must be an integer. Since $n > 0$, $n + 4 > 4$.
The factors of $35$ greater than $4$ are $5$, $7$ and $35$. This shows that there are three values for $n$:
\begin{itemize}
    \item $n + 4 = 5 \Rightarrow n = 1$
    \item $n + 4 = 7 \Rightarrow n = 3$
    \item $n + 4 = 35 \Rightarrow n = 31$
\end{itemize}

Therefore, the only positive integers, $n$, such that $\frac{n^2 + 8n + 51}{n + 4}$ is a positive integer are $n \in \{1, 3, 31\}.$

\vspace{6.81mm}

\item[3.] % Standard
\textit{Prove that for all real numbers $x, y$ and $z$,
$$x^2 + 5y^2 + z^2 \ge 2y(2x + z)$$}

\textit{Solution}:
$$\Rightarrow x^2 + 5y^2 + z^2 \ge 2y(2x + z)$$
$$\Rightarrow x^2 + 5y^2 + z^2 \ge 4xy + 2yz$$
$$\Rightarrow x^2 + 5y^2 + z^2 - 4xy - 2yz \ge 0$$
$$\Rightarrow x^2 - 4xy + 4y^2 + y^2 - 2yz + z^2 \ge 0$$
$$\Rightarrow (x - 2y)^2 + (y - z)^2 \ge 0$$

Squares of real numbers are never negative, so the sum of two squares of real numbers is greater than or equal to 0.\\

\vspace{6.81mm}

\item[4.] % Source
\textit{}

\textit{Solution}:
\vspace{6.81mm}

\item[5.] % Swiss Maths Olympiad Final round 2017
\textit{The cells of an $8 \times 8$ chessboard are all coloured in white. A move consists in inverting the colours of a $1 \times 3$ rectangle, either vertical or horizontal (the white cells become black and the black cells become white).
Is it possilbe to colour all cells of the chessboard in black in a finite number of moves?}

\textit{Solution}: 


\end{enumerate}

\end{document}
