\documentclass{article}

\usepackage{mathtools,amsfonts}
\usepackage{enumerate}
\usepackage{amsmath}
\usepackage{fullpage}


\begin{document}

\begin{center}
  \textbf{\Large Intermediate Test 2 Solutions}
  \\ \vspace{1em}
  \textbf{\large Stellenbosch Camp 2019}
\end{center}


\begin{enumerate}[1.]

\item[1.] % Source
\textit{In triangle $\triangle ABC$, the angle bisector of $\angle BAC$, the perpendicular bisector of $AC$ and the altitude from $C$ to $AB$ are concurrent. Find the value of $\angle BAC$.}

\textit{Solution}:
\\
Let the given altitude, angle bisector and perpendicular bisector intersect in $D$, let the foot of the perpendicular from $C$ to $AB$ be $E$ and let the midpoint of $AC$ be $F$. Since $AF$ = $FC$, $\angle AFE$ = $\angle CFE$ = $90^{\circ}$
and $FE$ is a common side, $\triangle EFC$ is congruent to $\triangle EFA$. Therefore $\angle ECF$ = $\angle EAF $ = $\angle EAD$. Furthermore, since $\angle CDA$ is a right angle, we have that $90^{\circ}$ =  $\angle DCA$ + $\angle DAC$ = 3$\angle DAE$, which gives $\angle DAE$ = $30^{\circ}$. Thus $\angle BAC$ = 2$\angle DAE$ = $60^{\circ}$.

\item[2.] % Standard
\textit{Find all positive integers $n$ such that $\frac{n^2 + 8n + 51}{n + 4}$ is also a positive integer.}

\textit{Solution}:
$$\frac{n^2 + 8n + 51}{n + 4} = n + \frac{n^2 + 8n + 51 - n(n + 4)}{n + 4}$$
$$\implies \frac{n^2 + 8n + 51}{n + 4} = n + \frac{4n + 51}{n + 4}$$
$$\implies \frac{n^2 + 8n + 51}{n + 4} = n + 4 + \frac{4n + 51 - 4(n + 4)}{n + 4}$$
$$\implies \frac{n^2 + 8n + 51}{n + 4} = n + 4 + \frac{35}{n + 4}$$

This shows that if $\frac{n^2 + 8n + 51}{n + 4}$ is a positive integer, then $\frac{35}{n + 4}$ must be an integer. Since $n > 0$, $n + 4 > 4$.
The factors of $35$ greater than $4$ are $5$, $7$ and $35$. This shows that there are three values for $n$:
\begin{itemize}
    \item $n + 4 = 5 \implies n = 1$
    \item $n + 4 = 7 \implies n = 3$
    \item $n + 4 = 35 \implies n = 31$
\end{itemize}

Therefore, the only positive integers, $n$, such that $\frac{n^2 + 8n + 51}{n + 4}$ is a positive integer are $n \in \{1, 3, 31\}.$


\item[3.] % Standard
\textit{Prove that for all real numbers $x, y$ and $z$,
$$x^2 + 5y^2 + z^2 \ge 2y(2x + z)$$}

\textit{Solution}:
$$\implies x^2 + 5y^2 + z^2 \ge 2y(2x + z)$$
$$\implies x^2 + 5y^2 + z^2 \ge 4xy + 2yz$$
$$\implies x^2 + 5y^2 + z^2 - 4xy - 2yz \ge 0$$
$$\implies x^2 - 4xy + 4y^2 + y^2 - 2yz + z^2 \ge 0$$
$$\implies (x - 2y)^2 + (y - z)^2 \ge 0$$

Squares of real numbers are never negative, so the sum of two squares of real numbers is greater than or equal to 0.



\item[4.] % Source
\textit{}

\textit{Solution}:


\item[5.] % Swiss Maths Olympiad Final round 2017
\textit{The cells of an $8 \times 8$ chessboard are all coloured in white. A move consists in inverting the colours of a $1 \times 3$ rectangle, either vertical or horizontal (the white cells become black and the black cells become white).
Is it possible to colour all cells of the chessboard in black in a finite number of moves?}

\textit{Solution}: 


\end{enumerate}

\end{document}
