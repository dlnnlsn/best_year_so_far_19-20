\documentclass{article}

\usepackage{mathtools,amsfonts}
\usepackage{enumerate}
\usepackage{amsmath}
\usepackage{fullpage}

\begin{document}

\begin{center}
  \textbf{\Large Intermediate Test 1 Solutions}
  % LEVEL is Advanced, Intermediate or Beginner
  % NUMBER is the test number: 1, 2, etc.
  \\ \vspace{1em}
  \textbf{\large Stellenbosch Camp 2019}
\end{center}


\begin{enumerate}[1.]

\item[1.] % 2009 Stellenbosch Camp (Junior Section, Test 3)
\textit{In triangle $\triangle ABC$, the angle bisector of $\angle BAC$, the perpendicular bisector of $AC$ and the altitude from $C$ to $AB$ are concurrent. Find the value of $\angle BAC$.}

\textit{Solution}:
Let the three lines intersect at $P$, the midpoint of $AC$ be $M$ and the foot of the altitude from $C$ onto $AB$ be $D$. Notice that $\triangle PAM \equiv \triangle PBM$, so $\angle PAM = \angle PBM$. Since $AP$ is the angle bisector of $\angle BAC$, $\angle PAM = \angle PAD$. Summing up the angles in $\triangle ABD$ gives:
$$\angle ABD + \angle BDA + \angle DAB = 180^o$$ 
$$\Rightarrow \angle PBA + 90^o + 2 \times \angle PBA = 180^o$$
$$\Rightarrow \angle PBA = 30^o$$
$$\Rightarrow \angle BAC = 30^o \times 2 = 60^o$$
\vspace{6.81mm}

\item[2.] % Standard
\textit{Find all positive integers $n$ such that $\frac{n^2 + 8n + 51}{n + 4}$ is also a positive integer.}

\textit{Solution}:
$$\frac{n^2 + 8n + 51}{n + 4} = n + \frac{n^2 + 8n + 51 - n(n + 4)}{n + 4}$$
$$\Rightarrow \frac{n^2 + 8n + 51}{n + 4} = n + \frac{4n + 51}{n + 4}$$
$$\Rightarrow \frac{n^2 + 8n + 51}{n + 4} = n + 4 + \frac{4n + 51 - 4(n + 4)}{n + 4}$$
$$\Rightarrow \frac{n^2 + 8n + 51}{n + 4} = n + 4 + \frac{35}{n + 4}$$

This shows that if $\frac{n^2 + 8n + 51}{n + 4}$ is a positive integer, then $\frac{35}{n + 4}$ must be an integer. Since $n > 0$, $n + 4 > 4$.
The factors of $35$ greater than $4$ are $5$, $7$ and $35$. This shows that there are three values for $n$:
\begin{itemize}
    \item $n + 4 = 5 \Rightarrow n = 1$
    \item $n + 4 = 7 \Rightarrow n = 3$
    \item $n + 4 = 35 \Rightarrow n = 31$
\end{itemize}

Therefore, the only positive integers, $n$, such that $\frac{n^2 + 8n + 51}{n + 4}$ is a positive integer are $n \in \{1, 3, 31\}.$

\vspace{6.81mm}

\item[3.] % Standard
\textit{Prove that for all real numbers $x, y$ and $z$,
$$x^2 + 5y^2 + z^2 \ge 2y(2x + z)$$}

\textit{Solution}:
$$\Rightarrow x^2 + 5y^2 + z^2 \ge 2y(2x + z)$$
$$\Rightarrow x^2 + 5y^2 + z^2 \ge 4xy + 2yz$$
$$\Rightarrow x^2 + 5y^2 + z^2 - 4xy - 2yz \ge 0$$
$$\Rightarrow x^2 - 4xy + 4y^2 + y^2 - 2yz + z^2 \ge 0$$
$$\Rightarrow (x - 2y)^2 + (y - z)^2 \ge 0$$

Squares of real numbers are never negative, so the sum of two squares of real numbers is greater than or equal to 0.\\

\vspace{6.81mm}

\item[4.] % Irish 2018 Q6, G
\textit{The points $E$ and $F$ lie on sides $AB$ and $AD$, respectively, of a parallelogram $ABCD$ such that $|AB| = 4|AE|$ and $|AD| = 4|AF|$.
Prove that $BF$, $DE$, and $AC$ are concurrent.}

\textit{Solution}:
Let $DE$ and $FB$ intersect at $G$. Join $BD$ and $AG$ and let $AG$ extended intersect $BD$ at $O$.

By Ceva's Theorem in $\triangle ABD$ we have
$$1 = \frac{DO}{OB} \cdot \frac{BE}{EA} \cdot \frac{AF}{FD} = \frac{DO}{OB} \cdot \frac{3}{1} \cdot \frac{1}{3}$$
hence $DO = OB$. Since the diagonals of a parallelogram bisect each other, $O$ lies on the diagonal $AC$, hence $BF$, $DE$ and $AC$ concurrent.
\vspace{6.81mm}

\item[5.] % Swiss Maths Olympiad Final round 2017
\textit{The cells of an $8 \times 8$ chessboard are all coloured in white. A move consists in inverting the colours of a $1 \times 3$ rectangle, either vertical or horizontal (the white cells become black and the black cells become white).
Is it possilbe to colour all cells of the chessboard in black in a finite number of moves?}

\textit{Solution}: 


\end{enumerate}

\end{document}
