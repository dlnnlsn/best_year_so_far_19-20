\documentclass{article}

\usepackage{mathtools,amsfonts}
\usepackage{enumerate}

\begin{document}

\begin{center}
  \textbf{\Large Intermediate Test 5 Solutions}
  \\ \vspace{1em}
  \textbf{\large Stellenbosch Camp 2019}
\end{center}


\begin{enumerate}[1.]

\item %Source of problem
\textit{Problem statement}

\textit{Solution}: Problem solution

\item %Source of problem
\textit{Problem statement}

\textit{Solution}: Problem solution

\item % Source 2001 booklet, The Monthly Problem Sets number 4
\textit{For each positive integer $k$, define the sequence $(a_{n})$ by
	\begin{align*}
	a_{0} &= 1 \\
	a_{n} &= kn + (-1)^{n}a_{n-1} \quad \text{for each } n \geq 1.
	\end{align*}
	Determine all values of $k$ for which 2000 is a term of the sequence.}

\textit{Solution}:
We prove that the terms in the sequence have the form
\begin{align*}
a_{4m} &= 4mk + 1 \\
a_{4m+1} &= k-1 \\
a_{4m+2} &= (4m + 3)k - 1 \\
a_{4m+3} &= 1
\end{align*}
We start by considering the odd terms in the sequence. If $n$ is odd then, from the definition of the sequence, 
\begin{align*}
a_{n+2} &= k(n + 2) + (-1)^{n+2}a_{n+1}\\
&= k(n + 2) - a_{n+1}\\
&= k(n + 2) - [k(n + 1) + (-1)^{n+1}a_{n}]\\
&= k(n + 2) - k(n + 1) - a_{n}\\
&= k - a_{n}
\end{align*}
Applying this formula twice, we have $a_{n+4} = k - a_{n+2} = a_{n}$. Since $a_{1} = k - 1$, the odd terms are given by $a_{4m+1} = k - 1$ and $a_{4m+3} = k - (k - 1) = 1$. 
At this point one can apply the definition of the sequence to confirm the formula given for the even terms.\\
There are thus four cases to consider to determine whether 2000 appears:
\begin{itemize}
	\item $2000 = 4mk + 1$, which has no integer solutions
	\item $2000 = k - 1$, which has 2001 as the only solution
	\item $2000 = (4m + 3)k - 1$, or $k = \frac{2001}{4m+3}$. The factors of 2001 are 1, 3, 23, 29, 69, 87, 667 and 2001 (since $2001 = 3 \times 23 \times 29$). Only 3, 23, 87 and 667 have the form $4m + 3$.
	\item $ 2001 = 1$, which clearly has no solutions
\end{itemize}
So the values of $k$ for which 2000 is a term is 3, 23, 87, 667 and 2001.

\item %Source of problem
\textit{Problem statement}

\textit{Solution}: Problem solution

\item %Source of problem
\textit{Problem statement}

\textit{Solution}: Problem solution

\item %Source of problem
\textit{Problem statement}

\textit{Solution}: Problem solution

\end{enumerate}

\end{document}
