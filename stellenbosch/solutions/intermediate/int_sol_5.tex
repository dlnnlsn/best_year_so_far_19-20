\documentclass{article}

\usepackage{mathtools,amsfonts}
\usepackage{enumerate}
\usepackage{amssymb}
\usepackage{amsthm}

\begin{document}

\begin{center}
  \textbf{\Large Intermediate Test 5 Solutions}
  \\ \vspace{1em}
  \textbf{\large Stellenbosch Camp 2019}
\end{center}


\begin{enumerate}[1.]

\item % Emile, C
\textit{Given the smiley face colouring, can the board be made completely white through some order of inverting rows.
}
	
\textit{Solution}:
Let every white squre be denoted by a $1$, and every black square by a $-1$. Let $k$ be the product of all of the values in the grid. 
Note that an inversion of a row or column would never change the value of $k$, as you are multiplying each item in the row by $(-1)$ when you invert. Hence, $k$ gets multiplied by $(-1)^8 = 1$.
	
Note that in the original diagram, $k = -1$ and a completely white board will have $k = 1$. Since inversions do not change the value of $k$ for the board, it must be impossible.
	
\vfill

\item % , 2018 December Monthly Assignment Q1
\textit{Let $n$ be a positive integer greater than 2. Let $r_1$ be the smallest odd divisor of $n$ greater than $1$ and let $r_2$ be the largest odd divisor of $n$. Find all $n$ such that
\begin{center}
 $n=5r_{1}+3r_{2}$
\end{center}}

\textit{Solution}: Problem solution

\vfill

\item % Source 2001 booklet, The Monthly Problem Sets number 4
\textit{For each positive integer $k$, define the sequence $(a_{n})$ by
	\begin{align*}
	a_{0} &= 1 \\
	a_{n} &= kn + (-1)^{n}a_{n-1} \quad \text{for each } n \geq 1.
	\end{align*}
	Determine all values of $k$ for which 2000 is a term of the sequence.}

\textit{Solution}:
We prove that the terms in the sequence have the form
\begin{align*}
a_{4m} &= 4mk + 1 \\
a_{4m+1} &= k-1 \\
a_{4m+2} &= (4m + 3)k - 1 \\
a_{4m+3} &= 1
\end{align*}
We start by considering the odd terms in the sequence. If $n$ is odd then, from the definition of the sequence, 
\begin{align*}
a_{n+2} &= k(n + 2) + (-1)^{n+2}a_{n+1}\\
&= k(n + 2) - a_{n+1}\\
&= k(n + 2) - [k(n + 1) + (-1)^{n+1}a_{n}]\\
&= k(n + 2) - k(n + 1) - a_{n}\\
&= k - a_{n}
\end{align*}
Applying this formula twice, we have $a_{n+4} = k - a_{n+2} = a_{n}$. Since $a_{1} = k - 1$, the odd terms are given by $a_{4m+1} = k - 1$ and $a_{4m+3} = k - (k - 1) = 1$. 
At this point one can apply the definition of the sequence to confirm the formula given for the even terms.\\
There are thus four cases to consider to determine whether 2000 appears:
\begin{itemize}
	\item $2000 = 4mk + 1$, which has no integer solutions
	\item $2000 = k - 1$, which has 2001 as the only solution
	\item $2000 = (4m + 3)k - 1$, or $k = \frac{2001}{4m+3}$. The factors of 2001 are 1, 3, 23, 29, 69, 87, 667 and 2001 (since $2001 = 3 \times 23 \times 29$). Only 3, 23, 87 and 667 have the form $4m + 3$.
	\item $ 2001 = 1$, which clearly has no solutions
\end{itemize}
So the values of $k$ for which 2000 is a term is 3, 23, 87, 667 and 2001.

\vfill

\item % Moldova 2018: 7.8, G
Let $\triangle XYZ$ be such that $\angle XZY = 30^o$. Let $M$ be a point inside $\triangle XYZ$. Let $A$ and $B$ be points on $XZ$ and $YZ$ respectively such that $\angle ZAM = \angle ZBM = 90^o$. Prove that $ZM = 2 \cdot AB$.

\textit{Solution}:
Denote by $O$ the midpoint of $ZM$. Notice then that $AMBZ$ is cyclic since $\angle ZAM = \angle ZBM = 90^\circ$. Also, $ZM$ is the diameter of the circle $AMBZ$ and $O$ is its centre. So then $AO=OB$ and $\angle AOB=2\angle AZB=60^\circ \implies \triangle AOC$ is equilateral. And so $AB=AO=MO=\frac{ZM}{2}$ which is what we wanted. 

\vfill

\item %The Malwande
\newcommand{\QQ}{\mathbb{Q}}
Find all functions $f : \QQ \to \QQ$ such that
\[ f(x^2) +f(x+2y) = (x+1)f(x) +2f(y) \]
for all $x, y \in \QQ$. 

\textit{Solution}: 
Let $x=y=0$ in the original equation to get that $f(0)+f(0)=f(0)+2f(0)$ and so $f(0)=0$. Next, letting $y=0$ gives that $f(x^2)+f(x)=(x+1)f(x) \implies f(x^2)=xf(x)$ for all $x \in \QQ$. Letting $x=0$ gives that $f(0)+f(2y)=f(0)+2f(y)$ for all $y \in \QQ$. The original equation therefore becomes $xf(x)+f(x+2y)=(x+1)f(x)+f(2y)$ and so $f(x+2y)=f(x)+f(2y)$ for all $x,y \in \QQ$ which is the Cauchy equation over the rationals, hence $f(x)=cx$ for some $c\in\QQ$. A straight-forward check shows this satisfies the original equation.

\vfill

\item % Austrian MO Q15, C
\textit{In the country of Oddland, there are stamps with values $1$ cent, $3$ cents, $5$ cents, etc., one type for each odd number.
The rules of Oddland Postal Services stipulate the following: for any two distinct values, the number of stamps of the higher value on an envelope must never exceed the number of stamps of the lower value.
In the country of Squareland, on the other hand, there are stamps with values $1$ cent, $4$ cents, $9$ cents, etc., one type for each square number.
Stamps can be combined in all possible ways in Squareland without additional rules.
Prove that for every positive integer $n$:
In Oddland and Squareland there are equally many ways to correctly place stamps of a total value $n$ cents on an envelope.
Rearranging the stamps on an envelope makes no difference.
}

\textit{Solution}: 
We will show that a bijection exists between Odd- and Squareland's systems.

Note: $n^2 = \sum_{i = 1}^{n}2i - 1$.

Let $a_1$, $a_2$, $a_3$, $\dots$, $a_s$ be the number of stamps of each denomination for Squareland where $a_i$ dictates the number of $i^2$ stamps.

Similarly, $b_1$, $b_2$, $\dots$, $b_t$ are the stamps for Oddland where $b_i$ dictates the number of $2i - 1$ stamps.

For any Squareland combination we have that:

\begin{align*}
	n &= \sum_ {i = 1}^{s} a_i \cdot i^2 \\
	&= \sum_ {i = 1}^{s} (a_i \sum_{j = 1}^{i} (2j - 1) ) \\
	&= \sum_ {j = 1}^{k} ((2j - 1) \sum_{i = 1}^{s - j + 1}a_i) \\
	&= \sum_ {j = 1}^{k} ((2j - 1) b_j 
\end{align*}
which is a valid Oddland combination as $a_i > 0 \;\forall\; i \in \mathbb{N}$.

The reverse also holds, meaning we have a bijection, and we are done.

\end{enumerate}

\end{document}
