\documentclass{article}

\usepackage{mathtools,amssymb}
\usepackage{enumerate}
\usepackage{geometry}
\usepackage{fullpage}


\title{Advanced Test 5}
\author{Stellenbosch Camp 2019}
\date{Time: $4$ hours}

\begin{document}
\maketitle
\thispagestyle{empty}

\hfill\textit{Each question is worth 7 marks.}

\begin{enumerate}[1.]

\item % Austrian MO Q15, C
\textit{In the country of Oddland, there are stamps with values $1$ cent, $3$ cents, $5$ cents, etc., one type for each odd number.
The rules of Oddland Postal Services stipulate the following: for any two distinct values, the number of stamps of the higher value on an envelope must never exceed the number of stamps of the lower value.
In the country of Squareland, on the other hand, there are stamps with values $1$ cent, $4$ cents, $9$ cents, etc., one type for each square number.
Stamps can be combined in all possible ways in Squareland without additional rules.
Prove that for every positive integer $n$:
In Oddland and Squareland there are equally many ways to correctly place stamps of a total value $n$ cents on an envelope.
Rearranging the stamps on an envelope makes no difference.
}

\textit{Solution}: 
We will show that a bijection exists between Odd- and Squareland's systems.

Note: $n^2 = \sum_{i = 1}^{n} 2i - 1$.

Let $a_1$, $a_2$, $a_3$, $\dots$, $a_s$ be the number of stamps of each denomination for Squareland where $a_i$ dictates the number of $i^2$ stamps.
Similarly, $b_1$, $b_2$, $\dots$, $b_t$ are the stamps for Oddland where $b_i$ dictates the number of $2i - 1$ stamps.
For any Squareland combination we have that:
\begin{align*}
  n &= \sum_ {i = 1}^{s} a_i \cdot i^2 \\
  &= \sum_ {i = 1}^{s} (a_i \sum_{j = 1}^{i} (2j - 1) ) \\
  &= \sum_ {j = 1}^{k} ((2j - 1) \sum_{i = 1}^{s - j + 1}a_i) \\
  &= \sum_ {j = 1}^{k} ((2j - 1) b_j 
\end{align*}
which is a valid Oddland combination as $a_i > 0 \;\forall\; i \in \mathbb{N}$.

The reverse also holds, meaning we have a bijection, and we are done.


\item % Turkish EGMO TST 2018 Q2, N
Find all pairs of positive integers $(m,n)$ satisfying
\[ m^2 +n^2 = 2019(m-n). \]

\textit{Solution}:
The given equation is equivalent to
\[
	(m + n)^2 + (m - n - 2019)^2 = 2019^2.
\]
Considering the equation modulo $3$, we determine that $m + n$ and $m - n - 2019$ must both be multiples of $3$, and thus $m$ and $n$ are both multiples of $3$. Let $m = 3a$ and $n = 3b$. We must now solve
\[
	(a + b)^2 + (a - b - 673)^2 = 673^2.
\]
The only possible common divisor of $(a + b)$ and $(a - b - 673)$ is $673$. If both are divisible by $673$, the equation reduces to
\[
	\left(\frac{a + b}{673}\right)^2 + \left(\frac{a - b - 673}{673}\right)^2 = 1
\]
which forces $a + b = 673$ and $a - b - 673 = 0$. But this gives us $n = 3b = 0$, which is not a valid solution.

We thus have that $(a + b)$ and $(a - b - 673)$ are relatively prime, and so $(a + b)$, $(a - b - 673)$, $673$ forms a primitive pythagorean triple. There are thus integers $x$ and $y$ such that either
\begin{align*}
	a + b & = 2xy & a - b - 673 & = x^2 - y^2 & 673 & = x^2 + y^2
\shortintertext{or}
	a + b & = x^2 - y^2 & a - b - 673 & = 2xy & 673 & = x^2 + y^2.
\end{align*}

The only way of writing $673$ as a sum of squares is $673 = 23^2 + 12^2$, and so $x = 23$ and $y = 12$, or $x = 12$ and $y = 23$. We thus have that $2xy = 552$ and $x^2 - y^2 = \pm 385$.

If $a + b = 552$ and $a - b - 673 = 385$, we get a negative value for $b$.
If $a + b = 552$ and $a - b - 673 = -385$, we get $a = 420$ and $b = 132$, which gives us $m = 1260$ and $n = 396$.
We can not have $a + b = -385$.
If $a + b = 385$ and $a - b - 673 = 552$, we also get a negative value for $b$.

Thus the only solution is $(m, n) = (1260, 396)$.

\item % The Ralph, C
Given a set of distinct points $(x_1, y_1)$, $(x_2, y_2)$, $\dots$, $(x_{8074}, y_{8074})$ with $x_i, y_i \in \{1, 2, \dots, 8076\}$.
Prove that there exists a positive integer $K$ such that there are $2019$ unique pairs $( (x_i, y_i), (x_j, y_j) )$ with
\[ |x_i - x_j| + |y_i - y_j| = K. \]

\textit{Solution}:
Notice that the number of possible Manhattan Distances in a $8076 \times 8076$ grid is $2 \times 8076 - 2 = 16150$. The total number of pairs of points is $\binom{8074}{2} = \frac{8074 \times 8073}{2} = 16150 \times 2018 + 1$.

Therefore, by the Pigeonhole Principle with 16150 distances as pigeonholes and the $\binom{8074}{2}$ pairs of points as pigeons, there will be at least $2019$ with distance $K$ apart, for some $K$.


\item % OME 2019 Q6, G
In a triangle $ABC$, the internal bisector of $\angle A$ meets the side $BC$ at $D$.
The lines through $D$ tangent to the circumcircles of triangles $\triangle ABD$ and $\triangle ACD$ meet the lines $AC$ and $AB$ at points $E$ and $F$, respectively.
Lines $BE$ and $CF$ intersect at $G$.
Prove that $\angle EDG = \angle ADF$.

\textit{Solution}:
Define $X= AD \cap EF, Y = DG \cap EF $.
Using a combination of vertically opposite angles and tan-chord theorem we obtain $\angle FDB = \angle FAB =\frac{1}{2} \angle BAC $ and $\angle EDC = \angle EAC =\frac{1}{2} \angle BAC $ and so $\angle EDF = 180^\circ -\frac{1}{2}\angle BAC -\frac{1}{2}\angle BAC= 180^\circ - \angle BAC$.
Thus $AEDF$ is a cyclic quadrilateral.
Moreover $\angle FED =\angle BAD = \angle DAC= \angle DEC$ so $EF \parallel CB$.
Thus considering $A$ and $G$ as centres of similarity we obtain 
\[
  \frac{EX}{XF} = \frac{CD}{DB} \quad \text{and} \quad \frac{EY}{YF} =\frac{BD}{DC}.
\]
It is well known that in $\triangle EDF $, $\angle FDX = \angle EDY \Leftrightarrow \frac{FX \cdot FY}{ EY \cdot EX} = \frac{FD^2}{ED^2}$.
Recalling that $ED = FD$ ($D$ is the midpoint of the arc $FDE$) and substituting in the above ratios the desired result follows.  


\item % The Andrew and The Dylan
\newcommand{\parens}[1]{\left(#1\right)}
Let $n$ be a positive integer greater than $1$, and consider a circle of radius $1$ in which is inscribed a regular $2n$-gon $P$ with vertices labelled from $1$ to $2n$ in that order.
Consider the set $S$ of positive divisors of $2n$, and the convex polygon $G$ formed by the points of $P$ with labels in $S$.
If the area of $G$ is denoted by $|G|$ and the number of elements of $S$ is denoted by $\tau(2n)$, show that
\[ \frac{1}{2} \parens{\sin\parens{\frac{\pi}{n}} +\cos\parens{\frac{\pi}{n}}} < |G| < \frac{\tau(2n)}{2} \sin\parens{\frac{\pi}{\tau(2n)}}. \]

\textit{Solution}:

Let the centre of the circle be $O$, then for successive divisors $d_k,d_{k+1}$ the area of the triangle $Od_kd_{k+1}$ is given by the sine area formula.
It follows that the area of $P$, $|G|$, is given by the sum 
\begin{equation}
  |G| = \sum_{k=0}^{m-1} \frac{1}{2}r^2 \sin \left( \frac{2 \pi(d_{k+1}-d_k)}{2n} \right) \label{eqn:series5}
\end{equation}
where $m=\tau(2n)$, $1=d_1<d_2<...<d_m=2n$ are the divisors of $2n$, $d_0=0$, and $r=1$ is the radius of the circle.
Notice that $d_{m-1}=n$, so the argument of $\sin$  in the last term in the sequence at \eqref{eqn:series5} is $\frac{2 \pi (2n - n)}{2n}= \pi $.
Thus for all $d_k<d_{m-1}$ we have that $\frac{2 \pi (d_{k+1} - d_k)}{2n} \leq \pi$.
We know that the sine function is concave on this interval and so we apply Jensen's Inequality to find an upper bound, but first we notice that $\sin(\pi) = 0 = \sin(0)$ so we replace the final term in \eqref{eqn:series5} with $\frac{1}{2}r^2\sin(0)$.
\begin{flalign*}
  |G| &= \sum_{k=0}^{m-2} \frac{1}{2}r^2 \sin \left( \frac{2 \pi(d_{k+1}-d_k)}{2n} \right) + \frac{1}{2}r^2 \sin(0) \\
  |G| &< m \left(\frac{1}{2}r^2 \sin \left( \frac{1}{m} \left( \sum_{k=0}^{m-2}\frac{2\pi(d_{k+1} - d_k)}{2n} +0 \right)  \right)  \right).
\end{flalign*}

The right hand side contains a telescoping series and thus simplifies to
\begin{flalign*}
  |G| &< \frac{m}{2}r^2 \sin \left( \frac{\pi\left( d_{m-1} - d_0 \right) }{n} \right) =\frac{\tau(2n)}{2}\sin \left( \frac{\pi}{\tau(2n)} \right) 
\end{flalign*}
as required.

For the left hand side we notice that $2 = d_2$ and $n = d_{m-1}$ are both divisors of $2n$ and thus the triangle $d_0d_2d_{m-1}$ is wholly contained within $P$.
Moreover $1 = d_1 | 2n$ and so the area is a strict lower bound.
The sine area formula yields its area as 
\begin{flalign*}
  |G|>|\triangle d_0d_2d_{m-1}| &= \frac{1}{2} r^2 \sin \left( \frac{\pi}{2n} (2-0) \right) + \frac{1}{2}r^2 \sin \left( \frac{\pi}{2n}(n-2) \right)   \\ 
  &= \frac{1}{2} \sin \left( \frac{\pi}{n}  \right) + \frac{1}{2} \sin \left( \frac{\pi}{2}-\frac{\pi}{n} \right)   \\ 
  &= \frac{1}{2} \sin \left( \frac{\pi}{n}  \right) + \frac{1}{2} \cos \left( \frac{\pi}{n} \right).
\end{flalign*}


\item % Dutch BxMO TST 2018 Q4, N
Do there exist a positive integer $k$ and a nonconstant sequence $a_1, a_2, a_3, \dotsc$ of positive integers such that for each positive integer $n$,
\[ a_n = \gcd(a_{n+k},a_{n+k+1})? \]

\textit{Solution}:

Such a $k$ and a sequence do not exist.
We prove this by contradiction, so suppose they do exist.
Note that $a_n | a_{n+k}$ and $a_n | a_{n+k+1}$ for all $n \ge 1$.
Using simple induction, it follows that $a_n | a_{n+\ell k}$ and $a_n | a_{n+\ell k+ \ell}$ for all $\ell \ge 0$.
We will prove by induction to $m$ that $ a_n | a_{n+mk+(m+1)}$ for all $- \le m \leq k - 1$.
For $m = k - 1$, this follows from $a_n | a_{n+k_k} =a_{n+(k-1)k+k}$.
Now suppose that for a certain $m$ with $1 \le m \le k - 1$ we have that $a_n | a_{n+mk+(m+1)}$. We also know that $a_n | a_{n+mk+m}$.
Therefore, as $a_{n+(m-1)k+m} = \gcd(a_{n+mk+m}, a_{n+mk+m+1})$, we also have $a_n | a_{n+(m-1)k+m}$.
This finishes the induction argument.

Substituting $m = 0$ yields $a_n | a_{n+1}$.
Because $a_n | a_{n+1}$, we also have $\gcd(a_n, a_{n+1}) = a_n$ for all $n$.
Hence, $a_n = a_{n-k}$ for all $n \ge k + 1$.
Now we have $a_{n-k} | a_{n-k+1} | a_{n-k+2} | \dotsb | a_n = a_{n-k}$.
Because these are all positive integers, $a_{n-k}, a_{n-k+1}, . . . , a_n$ must all be equal.
This must be true for all $n \ge k + 1$, hence the sequence is constant, which gives a contradiction.


\end{enumerate}


\end{document}
