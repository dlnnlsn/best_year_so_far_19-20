\documentclass{article}

\usepackage{mathtools,amssymb}
\usepackage[inline]{enumitem}
\usepackage{amsmath}
\usepackage{fullpage}


\begin{document}

\begin{center}
  \textbf{\Large Advanced Test 3 Solutions}
  \\ \vspace{1em}
  \textbf{\large Stellenbosch Camp 2019}
\end{center}

\vspace{12pt}


\begin{enumerate}

\item[1.] % Moldova 2018 8.6, A
{\itshape
Let $a$, $b$, and $c$ be positive real numbers such that $b, c \in [1,2)$ and
\[ \frac{a+b}{b(1+c)} +\frac{a+c}{c(1+b)} = 2. \]
Show that $a$, $b$, and $c$ are the lengths of the sides of a triangle. 
}

\textit{Solution}:
By multiplying out and simplifying the given equation, we get that $a = bc$.
Thus $a + b > a = bc \geq c$, and similarly $a+c > b$.
Finally,
\[ b+c > a = bc \iff 1 > 1 -b -c +bc = (1-b)(1-c), \]
which is true since $0 \leq b-1, c-1 < 1$.


\item[2.] % The Emile, C
{\itshape
You have $22$ points on a plane such that the perimeter of the triangle formed by any three of these points is at most $2$. Is it possible to cover all of the points in the plane with a strip of width $2\sqrt{2} - 2$?
}

\textit{Solution}:


\item[3.] % The Phil, N
{\itshape
Find the least positive integer $k$ such that $2050^{2051}$ can be written as a sum of $k$ $5$th powers.
}

\textit{Solution}:
By Fermat's Little Theorem: 
$$a^{10} \equiv _{11} 1 \; \forall a \in \mathbb{Z} \ \text{such that}\ \gcd(a, 11) = 1$$ 
$$\implies (a^5)^2 \equiv _{11} 1$$
$$\implies 11 \mid (a^5)^2 - 1$$
$$\implies 11 \mid (a^5 - 1)(a^5 + 1)$$
Since $11$ is prime, we must have $11 \mid a^5 - 1 \implies a^5 \equiv _{11} 1$ or $11 \mid a^5 + 1 \implies a^5 \equiv _{11} -1$. When $11 \mid a$, $a^5 \equiv _{11} 0$.
This shows that the only values of powers of $5$ mod $11$ are $-1$, $0$ and $1$. Notice that $2050^{2051} = (2050^{205})^{10} \times 2050 \equiv _{11} 4$ and the smallest way to express $4$ as the sum of some values from $\{-1, 0, 1\}$ is $4 = 1 + 1 + 1 + 1$.
This shows that $2050^{2051}$ cannot be expressed as the sum of less than $4$ fifth powers, so $k = 4$ is a lower bound.
$$(2050^{410} \times 2)^5 + (2050^{410} \times 2)^5 + (2050^{410} \times 1)^5 + (2050^{410} \times 1)^5 = (2050^{410 \times 5}) \times (2^5 + 2^5 + 1^5 + 1^5) = 2050^{2051}$$
This shows that $k = 4$ is possible, so $k = 4$ is the minimum.


\item[4.] % The Andrew, G
{\itshape
Consider an acute triangle $ABC$ with circumcircle $\Gamma$.
Let the tangents to $\Gamma$ at $B$ and $C$ intersect at a point $T$, the line $TA$ intersect $\Gamma$ again at $D$, and the point diametrically opposite $D$ with respect to $\Gamma$ be $E$.
Show that the angle bisector of $\angle BEC$ intersects $AT$ on a circle which is tangent to $BT$, $CT$, and $\Gamma$.
}

\textit{Solution}:


\item[5.] % PEN M1, N
{\itshape
Let $P$ be a polynomial with integer coefficients, and define a sequence $(a_n)$ by $a_0 = 0$ and $a_{n+1} = P(a_n)$ for $n \geq 0$.
Show that for nonnegative integers $m$ and $n$,
\[ \gcd(a_m,a_n) = a_{\gcd(m,n)}. \]
}

\textit{Solution}:


\end{enumerate}

\end{document}
