\documentclass{article}

\usepackage{mathtools,amsfonts}
\usepackage{enumerate}

\begin{document}

\begin{center}
  \textbf{\Large Advanced Test 1, Question 3 Solution}
  % LEVEL is Advanced, Intermediate or Beginner
  % NUMBER is the test number: 1, 2, etc.
  \\ \vspace{1em}
  \textbf{\large Stellenbosch Camp 2019}
\end{center}


\begin{enumerate}

\item[4.] Olimp\'iada Juvenil De Matem\'atica, Final Round K11, Problem 4.
\textit{Problem statement}

\textit{Solution}: Since $\triangle BAN \equiv \triangle CAN$ we have $\angle IAB = \angle NAB = \angle CAN = \angle CAI $, and thus $AI$ is an angle bisector of $\angle CAM$. Note that $\triangle AMC \equiv \triangle BMC$ we have $\angle AMC = \angle BMC$ and $\angle AMC + \angle BMC =180^\circ$, thus $\angle AMC = \angle BMC = 90^\circ$. Since $\angle AMC + \angle CPA = 90^\circ +90^\circ =180^\circ$ we have that AMCP is a cyclic quadrilateral. So $\angle AMI = \angle AMP = \angle ACP = \angle CAP = \angle CMP = \angle CMI$. Therefore $MI$ is a bisector of $\angle AMC$. This implies that $I$ is the incentre of $\triangle AMC$ and so $CI$ is the angle bisector of $\angle ACM$ by necessity.
\end{enumerate}

\end{document}
