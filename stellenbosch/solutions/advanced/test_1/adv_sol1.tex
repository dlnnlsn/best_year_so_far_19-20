\documentclass{article}

\usepackage{mathtools,amsfonts}
\usepackage{enumerate}
\usepackage{fullpage}


\begin{document}

\begin{center}
  \textbf{\Large Advanced Test 1 Solutions}
  \\ \vspace{1em}
  \textbf{\large Stellenbosch Camp 2019}
\end{center}


\begin{enumerate}[1.]

\item % Spanish MO Q1, C
{\itshape
A set $T$ of integers is called \emph{broken} if there are integers $a < b < c$ such that $a$ and $c$ are in $T$ but $b$ is not in $T$.

Find the number of broken subsets of $\{1, 2, \dotsc, 2019\}$.}

\textit{Solution}:
Problem solution


\item % Venezuela K11 Q4, G
{\itshape
Let $ABC$ denote an equilateral triangle.
Let $M$ and $N$ denote the midpoints of $AB$ and $BC$, respectively.
Let $P$ be a point outside $ABC$ such that $APC$ is isosceles and right-angled at $P$.
Lines $PM$ and $AN$ meet at $I$.
Prove that $CI$ is the angle bisector of $\angle ACM$.}

\textit{Solution:}
Since $\triangle BAN \equiv \triangle CAN$ we have $\angle IAB = \angle NAB = \angle CAN = \angle CAI $, and thus $AI$ is an angle bisector of $\angle CAM$. Note that $\triangle AMC \equiv \triangle BMC$ we have $\angle AMC = \angle BMC$ and $\angle AMC + \angle BMC =180^\circ$, thus $\angle AMC = \angle BMC = 90^\circ$. Since $\angle AMC + \angle CPA = 90^\circ +90^\circ =180^\circ$ we have that $AMCP$ is a cyclic quadrilateral. So $\angle AMI = \angle AMP = \angle ACP = \angle CAP = \angle CMP = \angle CMI$. Therefore $MI$ is a bisector of $\angle AMC$. This implies that $I$ is the incentre of $\triangle AMC$ and so $CI$ is the angle bisector of $\angle ACM$ by necessity.


\item % British MO 2013-2014, N
{\itshape
A number written in base $10$ is a string of $3^{2019}$ digit $3$s.
No other digit appears.
Find the highest power of $3$ which divides this number.}

\textit{Solution:}


\item % Irish 2018 Q7, C
{\itshape
There are $6$ eagles, $17$ snakes and $55$ mice in Wonderland.
An eagle can eat a snake or a mouse, but not another eagle.
A snake can eat a mouse, but not an eagle or another snake.
A mouse cannot eat an eagle, a snake or another mouse.

Whenever an eagle eats a snake, it turns into a mouse, and when it eats a mouse it turns into a snake. When a snake eats a mouse, it turns into an eagle.

After some time a situation is reached in which no animal can eat another animal.
What is the maximal possible number of animals alive in this situation?}

\textit{Solution:}
Note that the parity of the number of mice and number of snakes is always the same, whereas the parity of the number of eagles is always different; this is because the parity of each of these changes every time one animal eats another.
Also, the only possible situations wherein no animal can eat another is if there are only one kind of animal left, i.e. only mice, only eagles or only snakes.
Thus at the end we cannot have only mice left, since then the numbers of eagles and snakes would both be zero and the parities thus the same.
Similarly, we cannot have only snakes left, and so we must have only eagles left; we want to maximise the number of eagles left at the end.

Now after each step the number of mice decreases by at most one, so at least $55$ steps must take place before we reach the end.
Also, the total number of animals decreases by exactly one at each step, so the number of animals left at the end is at most $6 +17 +55 -55 = 23$.

Moreover, it is possible to have $23$ eagles left at the end, if the following eatings occur:
\begin{itemize}
  \item snake eats mouse $17$ times, leaving $23$ eagles, $0$ snakes, and $38$ mice;
  \item eagle eats mouse $19$ times, leaving $4$ eagles, $19$ snakes, and $19$ mice;
  \item snake eats mouse $19$ times, leaving $23$ eagles, $0$ snakes, and $0$ mice.
\end{itemize}


\item % USAMO 1993 Q1, A
{\itshape
For each integer $n \geq 2$, determine, with proof, which of the two positive real numbers $a$ and $b$ satisfying
\[ a^n = a+1, \qquad b^{2n} = b+3a \]
is larger.}

\textit{Solution:}


\end{enumerate}

\end{document}
