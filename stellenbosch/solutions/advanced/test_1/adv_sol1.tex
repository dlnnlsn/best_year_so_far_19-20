\documentclass{article}

\usepackage{mathtools,amsfonts}
\usepackage{enumerate}
\usepackage{fullpage}


\begin{document}

\begin{center}
  \textbf{\Large Advanced Test 1 Solutions}
  \\ \vspace{1em}
  \textbf{\large Stellenbosch Camp 2019}
\end{center}


\begin{enumerate}[1.]

\item % Spanish MO Q1, C
{\itshape
A set $T$ of integers is called \emph{broken} if there are integers $a < b < c$ such that $a$ and $c$ are in $T$ but $b$ is not in $T$.

Find the number of broken subsets of $\{1, 2, \dotsc, 2019\}$.}

\textit{Solution}:
The number of subsets of $\{1, 2, \dots, 2019\}$ is $2^{2019}$. Let us see how many of these subsets do not have the required property, that is, are not broken.
Clearly, neither the empty subset, nor any subset with just one element are broken. So at least
$$\binom{2019}{0} + \binom{2019}{1} = 1 + 2019 = 2020$$
subsets are not broken.\\
Suppose now that $T$ is a non-broken subset with two or more elements, and let $m$, $M$ be the smallest and largest elements in $T$, respectively. If some positive integer $k$ with $m \le k \le M$ is not in $T$, then $a = m$, $b = k$ and $c = M$ satisfy the broken condition, and $T$ would be broken, a contradiction; then all numbers $m, m + 1, m + 2, \dots, M$ must be in $T$, and any non-broken subset with more than two elements must be formed by consecutive numbers of $\{1, 2, \dots, 2019\}$.
The number of such subsets is $\binom{2019}{2}$, since all of them are determined by pairs of the form ($m$, $M$).\\
Then, the number of broken subsets $T$ of $\{1, 2, \dots, 2019\}$ is
$$2^{2019} - 2020 - \binom{2019}{2}$$


\item % Venezuela K11 Q4, G
{\itshape
Let $ABC$ denote an equilateral triangle.
Let $M$ and $N$ denote the midpoints of $AB$ and $BC$, respectively.
Let $P$ be a point outside $ABC$ such that $APC$ is isosceles and right-angled at $P$.
Lines $PM$ and $AN$ meet at $I$.
Prove that $CI$ is the angle bisector of $\angle ACM$.}

\textit{Solution:}
Since $\triangle BAN \equiv \triangle CAN$ we have $\angle IAB = \angle NAB = \angle CAN = \angle CAI $, and thus $AI$ is an angle bisector of $\angle CAM$. Note that $\triangle AMC \equiv \triangle BMC$ we have $\angle AMC = \angle BMC$ and $\angle AMC + \angle BMC =180^\circ$, thus $\angle AMC = \angle BMC = 90^\circ$. Since $\angle AMC + \angle CPA = 90^\circ +90^\circ =180^\circ$ we have that $AMCP$ is a cyclic quadrilateral. So
\[ \angle AMI = \angle AMP = \angle ACP = \angle CAP = \angle CMP = \angle CMI. \]
Therefore $MI$ is a bisector of $\angle AMC$. This implies that $I$ is the incentre of $\triangle AMC$ and so $CI$ is the angle bisector of $\angle ACM$ by necessity.


\item % British MO 2013-2014, N
{\itshape
A number written in base $10$ is a string of $3^{2019}$ digit $3$s.
No other digit appears.
Find the highest power of $3$ which divides this number.}

\textit{Solution:}
Let the number in question be $A$. Notice that $\frac{1}{3}A$ is a string of $3^{2019}$ $1$s. Consider more generally the number $B_n$ which consists of a string of $3^n$ digit $1$s.\\
Let $M_n$ be the number formed of a digit $1$, ($3^n - 1$) consecutive digits $0$, another digit $1$, another ($3^n - 1$) consecutive digits $0$ and then another $1$.
Notice that $B_n \times M_n = B_{n + 1}$.\\
Since the digital sum of $M_n$ is $3$, it is divisible by $3$. However, since this is not divisible by $9$, $M_n$ is not divisible by $9$. So $B_{n + 1}$ is divisible by exactly one higher power of $3$ than $B_n$.\\
Now $B_1 = 111$ is divisible by $3^1$ but not by $3^2$, and so $B_n$ is divisible by $3^n$ but not by $3^{n + 1}$. In particular, this means that $\frac{1}{3}A$ is divisible by $3^{2019}$, but by no higher power of $3$. Hence, $A$ is divisible by $3^{2020}$, but by no higher power of $3$.


\item % Irish 2018 Q7, C
{\itshape
There are $6$ eagles, $17$ snakes and $55$ mice in Wonderland.
An eagle can eat a snake or a mouse, but not another eagle.
A snake can eat a mouse, but not an eagle or another snake.
A mouse cannot eat an eagle, a snake or another mouse.

Whenever an eagle eats a snake, it turns into a mouse, and when it eats a mouse it turns into a snake. When a snake eats a mouse, it turns into an eagle.

After some time a situation is reached in which no animal can eat another animal.
What is the maximal possible number of animals alive in this situation?}

\textit{Solution:}
It is obvious that only one species can exist in this ending scenario. At the same time, we note that the combined number of any two species either decreases by two or remains unchanged after any transfiguration. This means that the parity of these combined numbers (snakes+eagles, eagles+mice, mice+snakes) remains invariant.
As two species need to disappear for the equilibrium state to be reached, their parity has to be even from the beginning. The only such pair is snakes+mice=72, implying that the remaining species will be eagles.
It also implies that all mice must perish, meaning that at most $6+17=23$ animals can survive. We can reach the state with $23$ eagles with the following procedure:
\begin{itemize}
  \item every snake eats a mouse: 23 eagles, 0 snakes, 38 mice
  \item $38/2=19$ eagles eat $19$ mice: $4$ eagles, $19$ snakes, $19$ mice
  \item every snake eats a mouse: $23$ eagles, $0$) snakes, $0$ mice.
\end{itemize}


\item % USAMO 1993 Q1, A
{\itshape
For each integer $n \geq 2$, determine, with proof, which of the two positive real numbers $a$ and $b$ satisfying
\[ a^n = a+1, \qquad b^{2n} = b+3a \]
is larger.}

\textit{Solution:}
$a^n = a+1 > 1 \implies a > 1$, and $b^{2n} = b+3a > b \iff b^{2n-1} > 1 \implies b > 1$.
Moreover,
\begin{align*}
  a^{2n} -b^{2n} &= (a+1)^2 -(b+3a) = (a-1)^2 +a -b > a-b \\
  \iff 0 &< a^{2n} -b^{2n} -a +b = (a-b)(a^{2n-1} +a^{2n-2}b +\dotsb +a b^{2n-2} +b^{2n-1} -1).
\end{align*}
Now $a^{2n-1} +a^{2n-2}b +\dotsb +a b^{2n-2} +b^{2n-1} -1 > a^{2n-1} -1 > 0$, so $a-b > 0$; thus $a$ is larger than $b$.


\end{enumerate}

\end{document}
