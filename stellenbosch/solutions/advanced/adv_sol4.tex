\documentclass{article}

\usepackage{mathtools,amssymb}
\usepackage[inline]{enumitem}
\usepackage{amsmath}
\usepackage{fullpage}


\begin{document}

\begin{center}
  \textbf{\Large Advanced Test 4 Solutions}
  \\ \vspace{1em}
  \textbf{\large Stellenbosch Camp 2019}
\end{center}

\vspace{12pt}


\begin{enumerate}

\item[1.] % Source Romania Junior 2018 Q1
\newcommand{\floorsqrt}[1]{\left\lfloor\sqrt{#1}\right\rfloor} 
\textit{
Find the positive integer solutions to the equation
\[ \floorsqrt{8n+1} +\floorsqrt{8n+2} +\dotsb +\floorsqrt{8n+7} = 2027. \]}

\textit{Solution}:
Notice that for $n > 1$, $\floorsqrt{8n + 1}$ and $\floorsqrt{8n + 7}$ differ by at most $1$. This means that all values $\floorsqrt{8n + 1}$, $\floorsqrt{8n + 2}$, $\dots$, $\floorsqrt{8n + 7}$ are either the same or differ by 1. Suppose that there are $a$ values that differ by $1$ and let $\floorsqrt{8n + 1} = k$.
$$7k + a = 2027$$
Notice that $2027 \equiv _7 4 \implies a = 4 \implies k = 289$ From this, we see that $\floorsqrt{8n + 1} = 289$ and specifically $\sqrt{8n + 4} = 290 \implies 8n + 4 = 290^2$. Therefore, the only value that works is $n = \frac{290^2 - 4}{8} = \frac{145^2 - 1}{2}$ which is an integer.

\item[2.] % Source Irish 2018 Q5
\textit{
Let $x_0, x_1,..., x_n$ be real numbers and define
\[y_k=x_k-x_{n-k}, \quad k=0,1,...,n.\]
Prove that 
\[y_0^2 + y_1^2+...+ y_n^2 \leq 4(x_0^2 + x_1^2 + ... + x_n^2) \]
and determine when equality holds.
}

\textit{Solution}: 
Substitute the definition of all $y_i$ into the inequality:
\begin{align*}
  &&\sum_{i = 0}^{n}y_i^2 &\leq 4\sum_{i = 0}^{n}x_i^2 &\\
  &\iff& \sum_{i = 0}^{n}(x_i - x_{n-i})^2 &\leq 4\sum_{i = 0}^{n}x_i^2 &\\
  &\iff& 2\sum_{i = 0}^{n}x_i^2 - 2\sum_{i = 0}^{n}x_ix_{n-i} &\leq 4\sum_{i = 0}^{n}x_i^2& \\
  &\iff& 0 &\leq 2\sum_{i = 0}^{n}x_i^2 + 2\sum_{i = 0}^{n}x_ix_{n-i} \\
  &\iff& 0 &\leq \sum_{i = 0}^{n}(x_i + x_{n-i})^2 &
\end{align*}

Since the square of a real number is always non-negative, the sum on the right must also be non-negative.
Equality only holds when
$$x_i + x_{n-i} = 0$$
for all choices of $i$ with $0 \le i \le n$.


\item[3.] % The Malwande, G
\textit{Let $ABC$ be a triangle, and let $M$ and $N$ be the midpoints of $AB$ and $CB$ respectively.
Let $T$ be the intersection of the line through $M$ perpendicular to $AC$ and the line through $B$ perpendicular to $BC$.
Show that $TN$ is equal to the radius of the circumcircle of $\triangle ABC$.}

\textit{Solution}:

Let the line perpendicular to $AC$ at $A$ intersect $TB$ at $F$.
By the midpoint theorem in $\triangle ABF$, we get $TB=TF$ since $AF\parallel MT$ and $MB=MA$.
Now by the midpoint theorem in $\triangle BFC$, we have that $CF=2NT$.
We also have that $CF$ is the diameter of the circumcircle of $\triangle BAC$ since $\angle CAF=90^\circ$.
It therefore follows that $NT$ is equal to the radius of the circumcircle of $\triangle ABC$ since it is half the diameter.


\item[4.] % Spanish MO Q2, N
\textit{Is there a finite set $S$ of positive integers, each of which is greater than $1$, such that for every positive integer $n$ greater than $3$ we have that $3^3 +4^3 +\dotsb +n^3$ is divisible by one of the values in $S$?}

\textit{Solution}:
Let $M = 3^3 + 4^3 + \dots + n^3 = \left(\frac{n(n + 1)}{2}\right)^2 - 9$.
Assume there exists some set $S$ such that the condition holds.
Note that if a composite number in $S$ divides $M$, its prime factors also divide $M$.
This means that we can consider $S$ to be a set of prime numbers.

Let the primes in $S$ be $p_1$, $p_2$, $\dots$, $p_s$ and let $m = \prod_{p_i \neq 3}p_i$.
There exists some value $k$ such that $km \equiv _3 2$.
Consider what happens when $n = 2km$.

For every value of $p_i$:
\begin{align*}
  && M &\equiv _{p_i} (km(2km + 1))^2 - 9 \\
  &\implies& M &\equiv _{p_i} 0 - 9 \equiv_{p_i} -9 \\
  &\implies& p_i &\nmid M &
\end{align*}
This shows that only possible value in the set that can divide $M$ is $3$.
However, since $n$ was constructed to have $n \equiv _3 1$:
\begin{align*}
  &&M &\equiv _3 (km(2km + 1))^2 - 9 \\
  &\implies& M &\equiv _3 (2 \times (1 + 1))^2 - 9 \\
  &\implies& M &\equiv _3 (4)^2 - 9 \\
  &\implies& M &\equiv _3 1 \\
  &\implies& 3 &\nmid M &
\end{align*}

This shows that for any finite set $S$, a value of $n$ can be constructed such that $3^3 + 4^3 + \dots + n^3$ is not divisible by any value in the set.
Therefore, no such set exists.


\item[5.] % The Jon, C
\textit{\newcommand{\CC}{\mathcal{C}}
Let $n$ be a positive integer, and let $\CC$ be a collection of subsets of $\{1, 2, \dotsc, 2n\}$ such that for every two distinct subsets $S_1, S_2 \in \CC$ neither of them is a subset of the other.
What is the maximal number of sets in $\CC$?}

\textit{Solution}:


\end{enumerate}

\end{document}
