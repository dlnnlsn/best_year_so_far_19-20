\documentclass{article}

\usepackage{mathtools,amsfonts}
\usepackage{enumerate}
\usepackage{amsmath}
\usepackage{fullpage}


\begin{document}

\begin{center}
  \textbf{\Large Advanced Test 2 Solutions}
  \\ \vspace{1em}
  \textbf{\large Stellenbosch Camp 2019}
\end{center}


\begin{enumerate}[1.]

\item[1.] % Irish 2018 Q6, G
\textit{The points $E$ and $F$ lie on sides $AB$ and $AD$, respectively, of a parallelogram $ABCD$ such that $|AB| = 4|AE|$ and $|AD| = 4|AF|$.
Prove that $BF$, $DE$, and $AC$ are concurrent.}

\textit{Solution}:
Let $DE$ AND $FB$ intersect at $G$.
Join $BD$ and $AG$ and let $AG$ extended intersect $BD$ at $O$.
By Ceva's Theorem in triangle $ABD$ we have
\[ 1 = \frac{DO}{OB} \cdot \frac{BE}{EA} \cdot \frac{AF}{FD} = \frac{DO}{OB} \cdot \frac{3}{1} \cdot \frac{1}{3}, \]
hence $DO = OB$.
Since the diagonals of a parallelogram bisect each other, $O$ lies on the diagonal $AC$; hence $BF$, $DE$, and $AC$ are concurrent.


\item[2.] % Swiss MO Q1, C
\textit{The cells of an $8 \times 8$ chessboard are all coloured in white.
A move consists in inverting the colours of a $1 \times 3$ rectangle, either vertical or horizontal (the white cells become black and the black cells become white).
Is it possible to colour all cells of the chessboard in black in a finite number of moves?}

\textit{Solution}:
Let us label the square in the $r$th row and $c$th column as $(r,c)$ where $1 \leq r,c \leq 8$.
Now let us colour the chessboard in three repeating diagonal colours red, green, and blue, where a square $(r,c)$ is red if $3 \mid r+c$, green if $3 \mid r+c-1$, and blue if $3 \mid r+c-2$.
The total number of red squares, which we denote as $R$, is $22$, and analogously $G = 21$ and $B = 21$.
We also keep track of the number of the number of white squares of each colour, which we denote by $R_o$, $G_o$, and $B_o$; initially these are also $22$, $21$, and $21$ respectively.

Note that after each move, the parities of each of $R_o$, $G_o$, and $B_o$ changes since one of each category is toggled by each $1 \times 3$ rectangle.
Thus $R_o$ and $G_o$ always have different parities, and in particular cannot both be zero.
Thus we cannot have all the squares of the chessboard be black.


\item[3.] % Ukraine MO 2018 First Tour Grade 9 Q3, N
\textit{Three numbers $2^{100}$, $3^{100}$, and $5^{100}$ are written on a long paper strip without any space in between, creating one big number $N$.
Ralpf claims that he can change the last digit of $N$ so that the new number is a power of $13$
Is he right?}

\textit{Solution}:


\item[4.] % Source
\textit{Find all functions $f: \mathbb{R} \to \mathbb{R}$ satisfying
\[ f(2xy) +f(f(x+y)) = xf(y) + yf(x) +f(x+y) \]
for all $x,y \in \mathbb{R}$.}

\textit{Solution}:


\item[5.] % Italian MO 2018, G
\textit{Let $\triangle ABC$ be acute and let $D$ be the foot of the perpendicular from $A$ onto $BC$.
The circle centred at $A$ passing through $D$ intersects the circumcircle of $\triangle ABC$ at $X$ and $Y$ (with $X$ on the same side as $B$ with respect to to the line $AD$).
Prove that $\angle BXD = \angle CYD$.}

\textit{Solution}: 


\end{enumerate}

\end{document}
