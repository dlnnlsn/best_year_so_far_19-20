\documentclass[11pt]{article}

\usepackage{mathtools,amssymb}
\usepackage{enumerate}
\usepackage{fancyvrb}
% \usepackage{fullpage}


\title{Test 1}
\author{`April' Camp 2020}
\date{Time: $4\frac{1}{2}$ hours}

\begin{document}
\maketitle
\thispagestyle{empty}


\begin{enumerate}[1.]

\vfill
\item % Malwande
Find all functions $f : \mathbb{R} \to \mathbb{R}$ such that
\[ f(x^2) -yf(y) = f(x+y)(f(x)-y) \]
for all real numbers $x$ and $y$.


\vfill
\item % IMO2019 G2
Let $ABC$ be an acute-angled triangle and let $D$, $E$ and $F$ be the feet of the altitudes from $A$, $B$, and $C$ to sides $BC$, $CA$, and $AB$ respectively.
Denote by $\omega_B$ and $\omega_C$ the incircles of triangles $BDF$ and $CDE$, and let these circles be tangent to segments $DF$ and $DE$ at $M$ and $N$ respectively.
Let line $MN$ meet circles $\omega_B$ and $\omega_C$ again at $P \neq M$ and $Q \neq N$ respectively.
Prove that $MP = NQ$.


\vfill
\item % IMO2019 C4
\newcommand{\LL}{\mathfrak{L}}
On a flat plane in Camelot, King Arthur builds a labyrinth $\LL$ consisting of $n$ walls, each of which is an infinite straight line.
No two walls are parallel, and no three walls have a common point.
Merlin then paints one side of each wall entirely red and the other side entirely blue.

At the intersection of two walls there are four corners: two diagonally opposite corners where a red side and a blue side meet, one corner where two red sides meet, and one corner where two blue sides meet.
At each such intersection, there is a two-way door connecting the two diagonally opposite corners at which sides of different colours meet.

After Merlin paints the walls, Morgana then places some knights in the labyrinth.
The knights can walk through doors, but cannot walk through walls.

Let $k(\LL)$ be the largest number $k$ such that, no matter how Merlin paints the walls of the labyrinth $\LL$, Morgana can always place $k$ knights such that no two of them can ever meet.
For each $n$, what are all possible values for $k(\LL)$, where $\LL$ is a labyrinth with $n$ walls?

\vfill
\end{enumerate}


\vfill
% ASCII art
\centering
\begin{BVerbatim}
\end{BVerbatim}

\end{document}
