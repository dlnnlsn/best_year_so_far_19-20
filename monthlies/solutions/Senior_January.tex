\documentclass{article}

\usepackage{amsmath,amssymb}
\usepackage{fullpage}
\usepackage{enumerate}
\usepackage{hyperref}
\usepackage{graphicx}
\graphicspath{{../logos/}}


\begin{document}

\setlength{\tabcolsep}{6pt}
\begin{center} \begin{tabular}{cccc}
	\includegraphics[height=56pt]{SAMF_logo.jpg} &
	\includegraphics[height=56pt]{SAICA_logo.jpg} &
	\includegraphics[height=56pt]{OM_Logo_Stacked_Vignette_on_White_RGB.jpg} &
	\includegraphics[height=56pt]{SAMO2019.png}
\end{tabular} \end{center}


\bigskip


\begin{center}
	\textbf{\Large <LEVEL> Monthly Problem Set Solutions}
\end{center}

\medskip
\begin{enumerate}

\item[1.] % PAMO Shortlist 2019
{\itshape On the board, we write the integers $1, 2, 3, \dots, 2019$.
At each minute, we pick two numbers on the board $a$ and $b$, erase them, and write down the number $s(a + b)$ instead where $s(n)$ denotes the sum of the digits of the integer $n$.
Let $N$ be the last number remaining on the board.
\begin{enumerate}
	\item Is it possible that $N = 19$?
	\item Is it possible that $N = 15$?
\end{enumerate}}

\medskip
\item[SOLUTION]

Since $10^{n} \equiv 1 \pmod 3$ for all non-negative integers $n$, we have that $s(n) \equiv n \pmod 3$ and so replacing $a$ and $b$ with $s(a+b)$ does not change the sum of the numbers on the board in mod 3. 
Now $1 + 2 +... + 2019 = \frac{(2019)(2020)}{2} \equiv 0 \pmod 3$ but $19 \not\equiv 0 \pmod 3$ so we cannot have $N=19$ as the last number on the board. 

We now show that we can in fact get to $15$. 

For each $k \neq \{1010, 906\}$, replace the numbers $k$ and $2020-k$ with $s(k + 2020-k) = s(2020)$ = $4$. So remaining on the board now is $906, 1114, 1010$ and $1008$ $4$s. Now taking the $4$s in pairs and applying the procedure gives $504$ $8$s, taking these in pairs gives $252$ $7$s, and taking these $7$s in pairs gives $126$ $5$s and again taking these $5$s as pairs gives $63$ $1$s. So we have left $906, 1114, 1010$ and $63$ $1$s. Now apply the procedure in $1114$ and $1010$ to get $s(2125)=9$ and remain with $9, 906$ and $63$ $1$s. Now from the $63$ $1$s, we can make $7$ $9$s by taking $1$s and applying the procedure until we get to a $7$ then we stop and take a $1$ that has not been used. We then have $8$ $9$s and $906$ left on the board. Since $s(9 + 9) = s(18) = 9$, we can apply this procedure to the $9$s until just one $9$ remains. Finally, we combine this $9$ with the $906$ to get $s(906 + 9) = s(915) = 15$.

Remark: {\itshape The steps used to get to 15 are not unique.}
\medskip
\item[2.]

\medskip
\item[3.] % Liam
Let $M$ be a positive integer, and let $S$ denote the set of finite sequences of positive integers less than or equal to $M$, including the empty sequence of length zero, which we denote as $\mathbf{0}$.
Also, for a sequence $\mathbf{x} = (x_1, x_2, \dotsc, x_n)$ let $\overline{\mathbf{x}}$ denote the reverse sequence $\overline{\mathbf{x}} = (x_n, x_{n-1}, \dotsc, x_1)$.

Define a function $d$ from $S$ to the integers as follows:
\begin{itemize}
\item $d(\mathbf{0}) = 0$.
\item If $\mathbf{x} = (x_1, x_2, \dotsc, x_n)$ is a sequence of positive length, let $m$ be the largest integer such that $x_1 +x_2 +\dotsb +x_m \leq M$, and let $\mathbf{x}'$ denote the rest of the sequence: $\mathbf{x}' = (x_{m+1}, \dotsc, x_n)$.
Then $d(\mathbf{x}) = 1 +d(\mathbf{x}')$.
\end{itemize}
Show that $d(\mathbf{x}) = d(\overline{\mathbf{x}})$.

\medskip
Note that $d$ partitions the sequence $\mathbf{x}$ into sub-sequences $s.t.$ the sum of each sub-sequence does not exceed $M$ and then counts the total amount of these sub-sequences. Let ${s_1, s_2, .., s_{d(\mathbf{x})}}$ be the indices of the  starts of these sub-sequences.

For the partitioning of $\bar{\mathbf{x}}$ we conjecture that none of the sub-sequences can have more than one of these $x_{s_i}$

If there are exist a sub-sequence with at least two of these $s_i$ we have:
\begin{center}
    $M \geq (\sum_{j=s_{i+1}}^{s_i} x_j) = (\sum_{j=s_i}^{s_{i+1}-1} x_j) + x_{s_{i+1}} > M$ 
\end{center}
which is a contradiction $\implies$ each $s_i$ in its own sub-sequence which implies $d(\overline{\mathbf{x}}) \geq d(\mathbf{x})$

Hence $d(\mathbf{x}) = d(\overline{\overline{\mathbf{x}}}) \geq d(\overline{\mathbf{x}}) \geq d(\mathbf{x})\implies d(x) = d(\bar x)$


\medskip
\item[4.]


\medskip
\item[5.]


\medskip
\item[6.]


\medskip
\item[7.]


\medskip
\item[8.]


\end{enumerate}

\end{document}
