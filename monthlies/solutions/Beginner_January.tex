\documentclass{article}

\usepackage{amsmath,amssymb}
\usepackage{fullpage}
\usepackage{enumerate}
\usepackage{hyperref}
\usepackage{graphicx}


\begin{document}

\begin{center}
	\textbf{\Large Beginner Monthly Problem Solutions}
	\\ \vspace{1em}
	\textbf{\large Due: Friday, 17 January 2020}
\end{center}

\begin{enumerate}[1.]

\vspace{6pt}
\item % 2010 Team Q6 
How many positive square numbers are factors of 1600?
\medskip
\item[ANS:]
Factorising yield: $1600 = 2^6 \times 5^2 = (2^2)^3 \times (5^2)^1 $
Note that a square number has only even exponents in its factorization, hence is made of square factors. Hence the square factors of $1600 = (3+1)(1+1) = 8$ 

\vspace{6pt}
\item % 2011 Team Q5
A palindrome is a positive integer whose digits are the same when read forwards or backwards.
For example, 2882 is a four-digit palindrome and 49194 is a five-digit palindrome. There are
pairs of four-digit palindromes whose sum is a five-digit palindrome. One such pair is 2882
and 9339. How many such pairs are there including this pair?

\medskip
\item[ANS:]
Let $P_1 = abba, P_2 = cddc$
Note that the first digit, and therefor the last digit, of the $5$ digit palindrome, $P_1+P_2 = xyzyx$, is $1 \implies x = 1 \implies a+c = 11 \implies y = 1,2$

If $y=1 \implies b+d < 10$ and $b+d \equiv_{10} = 0 \implies b+d = 0 \implies b=d=0$, hence there are $9-2+1 = 8$ possibilities in this case.
If $y=2 \implies b+d \geq 10$ and $b+d \equiv_{10} = 1 \implies b+d = 11$, hence there are $(9-2+1)\times(9-2+1) = 64$ possibilities in this case.

In total there are $72$ ordered possibilities $\implies 36$ pairs.

\vspace{6pt}
\item %2013 Team Q6
Show all the ways the number 365 can be written as the sum of two or more different perfect
square numbers?
\medskip
\item[ANS:]
$365 = a^2+b^2$ W.L.O.G $x>y$
Note we only have check for $y \leq 13$ as $\sqrt{\frac{365}{2}} < 14$
The only cases which work is $y = 2,13$ which has integer solutions for $x=19,14$ respectively.

\vspace{6pt}
\item %2014 Team Q5
How many squares on an $8 \times 8$ chessboard are more black than white?
\medskip
\item[ANS:]
Note that even $\times$ even squares have the same amount of black and white squares, hence we will only look at odd $\times$ odd squares. Note that the total amount of odd $\times$ odd squares has the same amount of more white as more black squares by symmetry. Therefor the amount of total odd $\times$ odd squares is twice the amount of more black squares. 
Total more black $ = \frac{2^2+4^2+6^2+8^2}{2} = 60$

\vspace{6pt}
\item %2014 Team Q8
In triangle $ABC$, $\widehat{A} = 45^{\circ}$. Point $P$ is on $BC$, $Q$ on $AB$ and $R$ on $AC$ such that $BP = QP$ and $CP = RP$. Find $Q \widehat{P} R$.
\medskip
\item[ANS:]

$B\hat{Q}P = \hat{B}$ 
$\implies B\hat{P}Q = 180 - 2\hat{B}$ \\
Similarly $C\hat{P}R = 180 - 2\hat{C}$ 
\begin{center}
$\implies Q\hat{P}R = 180 - (180 - 2\hat{C}) - (180 - 2\hat{B}) = 180 - 2(180 - \hat{C} - \hat{B}) = 180 - 2\hat{A} = 90$
\end{center}

\vspace{6pt}
\item % 2016 Team Q3
If $X$ and $Y$ are two digits and the five digit number $30XY5$ can be expressed as the product $225 \times n$, find all possible positive integer values of $n$.
\medskip
\item[ANS:]
$30XY5$ is odd, hence $225\times n$ is odd $\implies n$ is odd.
\begin{center}
$n = \lceil{\frac{30XY5}{225}}\rceil \geq \lceil{\frac{30005}{225}}\rceil = 134$ \\
$n = \lfloor{\frac{30XY5}{225}}\rfloor \leq \lfloor{\frac{30995}{225}}\rfloor = 137$ \\
$\implies n = 135,137$
\end{center}


\vspace{6pt}
\item %Junior 2009 Team Competition
The symbol $\left \lfloor{x}\right \rfloor$ means the greatest integer less than or equal to $x$. For example, 
\begin{center}
	$\left \lfloor{5.7}\right \rfloor = 5$ and $\left \lfloor{4}\right \rfloor = 4$.
\end{center} 
Calculate the value of the sum
\begin{center}
	$\left \lfloor{\sqrt{1}}\right \rfloor$ + $\left \lfloor{\sqrt{2}}\right \rfloor$ + $\left \lfloor{\sqrt{3}}\right \rfloor$ + $\left \lfloor{\sqrt{4}}\right \rfloor$ + ... + $\left \lfloor{\sqrt{49}}\right \rfloor$ + $\left \lfloor{\sqrt{50}}\right \rfloor$
\end{center}
\medskip
\item[ANS:]
Note $\left \lfloor{\sqrt{n^2}}\right\rfloor = n$ and $\left \lfloor{\sqrt{n^2-1}}\right\rfloor = n-1$ and  $\left \lfloor{\sqrt{x}}\right\rfloor$ is an increasing function.\\
Hence the sum becomes:\\
$1 + 1 + 1 + 2 + 2 + ... + 6 + 7 + 7 = 1(2^2-1^2) + 2(3^2-2^2) + 3(4^2-3^2) + ... + 6(7^2-6^2) + 7 + 7 = 217$


\vspace{6pt}
\item %2012 Team Q7
On a page, there are 10 circles. Any pair of them intersect at exactly 2 points but no 3 of
them intersect at one common point. Into how many parts do these circles divide the page?
\item[ANS:]
Let $C$ be one of the circles. $C$ meets each of the other $n-1$ circles in 2 points. The $2n-2$ points where $C$ crosses one of the other circles divide the circle $C$ into $2n-2$ arcs, which we consider as edges of a plane graph. There are $2n-2$ such arcs on each of the $n$ circles, making a total of $E=n(2n-2)$ edges in the graph

Now we can use the following formula:
\begin{center}
$F=E-V+2=n(2n-2)-n(n-1)+2=n(n-1)+2=n^2-n+2$
\end{center}
Where $F$ is the amount of parts, $V$ the amount of vertices and $E$ the amount of edges we just calculated.

Subbing in $n = 10$ yields $F = 100 - 10 + 2 = 92$


\vspace{6pt}
\item %2015 Team Q9
Note that by choosing any two points which aren't diametrically opposite, we can find the other two points which will form a rectangle by taking the two points which are diametrically opposite the two chosen points. Also note that for every rectangle consisting of 4 points, choosing any 2 points (except for 2 which are diametrically opposite) it will define the rectangle.

The total amount of rectangles is:
$ \frac{{12\choose2} - 6}{4} = 15$

\vspace{6pt}
\item %2012 Team Q5
What is the remainder when $6^{2012} + 8^{2021}$ is divided by 49?
\medskip
\item[ANS:]
$6^{2012} + 8^{2021} \equiv_{49} 6^{-4} + 8^{5}$ by Euler Phi\\
The multiplicative inverse of $6$ is $-8$ (mod $49$)\\
Hence the above simplifies to $(-8)^{4} + 8^{5}\equiv_{49} 29+36 \equiv_{49} 16$


\end{enumerate}

\vspace{8pt}
\textbf{\Large Email submission guidelines}
\begin{itemize}
	\item Email your solutions to \href{mailto:samf.training.assignments@gmail.com}{\texttt{samf.training.assignments@gmail.com}}.
	\item In the subject of your email, include your name and the level of the assignment (Beginner, Intermediate or Senior).
	\item Submit each question in a single separate PDF file (with multiple pages if necessary), with your name and the question number written on each page.
	\item If you take photographs of your work, use a document scanner such as Office Lens to convert to PDF.
	\item If you have multiple PDF files for a question, combine them using software such as PDFsam.
\end{itemize}

\end{document}
