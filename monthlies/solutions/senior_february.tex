\documentclass{article}

\usepackage{amsmath,amssymb}
\usepackage{fullpage}
\usepackage{enumerate}
\usepackage{hyperref}
\usepackage{graphicx}
\graphicspath{{../logos/}}


\begin{document}

\setlength{\tabcolsep}{6pt}
\begin{center} \begin{tabular}{cccc}
	\includegraphics[height=56pt]{SAMF_logo.jpg} &
	\includegraphics[height=56pt]{SAICA_logo.jpg} &
	\includegraphics[height=56pt]{OM_Logo_Stacked_Vignette_on_White_RGB.jpg} &
	\includegraphics[height=56pt]{SAMO2019.png}
\end{tabular} \end{center}


\bigskip


\begin{center}
	\textbf{\Large Senior February Monthly Problem Set Solutions}
\end{center}

\begin{enumerate}

\medskip
\item % Malwande
{\itshape Let $ABC$ be a triangle with orthocentre $H$, such that $AB<BC$ and $\angle BAC < 90^\circ$. Let the circle $\Gamma$ centred at $B$ and passing through $A$ intersect $AC$ again at $D$. The circumcircle of $\triangle BCD$ intersects $\Gamma$ again at $E$. $ED$ and $BH$ intersect at $F$. Prove that $BD$ is tangent to the circumcircle of $\triangle DHF$.
}
% TODO: complete


\medskip
\item % IMO Long List 1992 Q65, A
{\itshape For any 4 points in $\mathbb{R}^3$, does there exist a plane such that the orthogonal projections of the points on the plane make up the vertices of a parallelogram?

}


\medskip
\item % IMO Long List 1992 Q75, N
{\itshape Find all positive integers $n$ that can be written in the form:
$$n = \left\lfloor m + \sqrt{m} + \frac{1}{2} \right\rfloor$$
where $m$ is also a positive integer.}



\medskip
\item % A Russian Olympiad 1975, C
{\itshape An $8 \times 8$ chessboard is divided into several regions by 13 straight lines. Can the lines be placed in such a way that each region contains at most 1 centre of the original 64 squares?}




\medskip
\item % Adapted from Sharygin 2005, G11 P1, G
{\itshape Let the midpoints of the sides $BC$, $CA$, $AB$ of an equilateral triangle $ABC$ be $D$, $E$, and $F$ respectively. Let $j$, $k$, and $\ell$ be lines passing through $D$, $E$, and $F$ respectively such that $j \parallel k \parallel \ell$. Define points $P$, $Q$, and $R$ by $P = j\cap EF$, $Q = k \cap FD$, and $R = \ell \cap DE$. Prove that the points $X$, $Y$, and $Z$ defined by $X = BC \cap QR$, $Y = CA \cap RP$, and $Z = AB \cap PQ$ are  collinear.}



\medskip
\item % Mongolia 2008 TST Day 3 Q3
{\itshape Show that there are only finitely many solutions $(x, y) \in \mathbb{N}^2$ to the equation
$$\sum_{i = 1}^{m} (x + i)^n = \sum_{i = 1}^{m} (y + i)^{2n}$$
where $m, n \in \mathbb{N} \setminus \{1\}$ are given constants.}

By the Power-Mean Inequality, we have:
$$(\frac{(x + i)^n + (x + m + 1 - i)^n}{2})^{\frac{1}{n}} > x + \frac{m + 1}{2}$$
$$\implies \frac{(x + i)^n + (x + m + 1 - i)^n}{2} > (x + \frac{m + 1}{2})^n$$
$$\implies \sum_{i = 1}^{n} (x + i)^n > m(x + \frac{m + 1}{2})^n$$

Expanding the LHS gives:
$$\sum_{i = 1}^{m} (x + i)^n = mx^n + \frac{mn(m + 1)}{2}x^{n - 1} + O_1(x^{n - 2})$$

Similarly:

$$m(x + \frac{m + 2}{2})^n = mx^n + \frac{mn(m + 2)}{2}x^{n - 1} + O_2(x^{n - 2})$$

Since the coefficients of the second term will dominate the expression after subtracting the two, we have:

$$m(x + \frac{m + 2}{2})^n > \sum_{i = 1}^{m} (x + i)^n > m(x + \frac{m + 1}{2})^n ~\forall x > T$$

where $T$ is some sufficiently large real number. Following a similar process and replacing $n$ by $2n$ and $x$ by $y$, we get:

$$m(y + \frac{m + 2}{2})^{2n} > \sum_{i = 1}^{m} (y + i)^{2n} > m(y + \frac{m + 1}{2})^{2n} ~\forall y > K$$

where $K$ is some sufficiently large real number. Adding $1$ to $y$ yields:

$$m(y + \frac{m + 2}{2} + 1)^{2n} > \sum_{i = 1}^{m} (y + i + 1)^{2n} > m(y + \frac{m + 1}{2} + 1)^{2n} ~\forall y > K$$

Since $m$ is fixed, we can choose $y > K$ such that: 
$$y^2 + (m + 2)y + (\frac{m + 2}{2})^2 - \frac{m + 1}{2} > T$$

\medskip
\item % Spain, Round 2, 1992
{\itshape In the land of Graphopia there are $n$ towns. Between some pairs of towns there are direct roads. The residents of these towns want to have mathematics tournaments. They've decided that in order to have a mathematics tournament they need to have $a$ participating towns. Further they decide that in order to have a tournament the connections between the towns involved must be symmetrical. That is either every pair of towns in a tournament are directly linked by a road or no pair is. 

As mathematics competitions are rightly adored by every member of Graphopia every grouping of $a$ towns that can hold a tournament holds one once a year (some towns may be in multiple groupings). However in some years new roads are built and old ones lost to the elements. If a group of $a$ towns find themselves able to put on a tournament they immediately do. If a tournament finds itself unallowable due to road connections, that tournament is regretfully discontinued. Prove that it is possible that one year poor Graphopia finds itself having less than $\binom{n}{a} $$2^{1-\binom{a}{2}}$ tournaments.
}



\medskip
\item % Serbian Mathematical Olympiad 2007
{\itshape For positive real numbers $a$, $b$, and $c$ with $a+b+c=1$ prove that:
\begin{align*}
 	\frac{a^{n+2}}{a^{n+1} + b^n + c^n} + \frac{b^{n+2}}{a^n + b^{n+1} + c^n} + \frac{c^{n+2}}{a^n + b^n + c^{n+1}} \ge \frac{1}{7}
\end{align*}

where $n \in \mathbb{N}$. Where does equality occur?}


\end{enumerate}

\end{document}