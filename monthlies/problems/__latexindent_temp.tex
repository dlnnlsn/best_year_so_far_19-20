\documentclass{article}

\usepackage{amsmath,amssymb}
\usepackage{mathtools}
\usepackage{hyperref}
\usepackage{amssymb}
\usepackage{graphicx}
\usepackage{siunitx}
\graphicspath{{../logos/}}


\begin{document}

\setlength{\tabcolsep}{5pt}
\begin{center} \begin{tabular}{cccc}
	\includegraphics[height=43pt]{SAMF_logo.jpg} &
	\includegraphics[height=43pt]{SAICA_logo.jpg} &
	\includegraphics[height=43pt]{OM_Logo_Stacked_Vignette_on_White_RGB.jpg} &
	\includegraphics[height=43pt]{SAMO2019.png}
\end{tabular} \end{center}

\bigskip

\begin{center}
\textbf{\Large An Easy Senior Monthly Problem Set :)}
\\ \vspace{1em}
\textbf{\large Due: XX, XX XX 2020}
\end{center}

\begin{enumerate}

\medskip
\item % Emile Tredoux, G
Pythagoras has discovered pizza! And immediately proceeded to working on combining it with his one true love, rational numbers.

He devises a way to cut a circular pizza such that all cuts go through the center of the pizza and that the ratio of areas of adjacent slices of pizza are rational numbers. He calls such a pizza a \textit{rational} pizza.

Prove that all \textit{rational} pizzas have rational internal angles at the center of the pizza, as this will make Pythagoras very happy.

\medskip
\item  %

\medskip
\item % Emile Tredoux, G
Given a triangle, $\triangle ABC$, with $AB = BC$ and circumcenter $O$, let $C'$ be a point on $AB$ such that $AC = AC'$ with $B$ and $C$ on the same side as $A$. Let the circle through $CBC'$ intersect with line $OB$ at $P$, and let $P'$ be the point diametrically opposite $P$. Finally let $Q$ and $Q'$ be the altitudes dropped from $P$ and $P'$ onto $BC$ respectively. Prove $QQ' = BC'$.

\medskip
\item % 

\medskip
\item % 

\medskip
\item % 

\medskip
\item % 

\medskip
\item % 
\end{enumerate}


\vfill
\textbf{\Large Email submission guidelines}
\begin{itemize}
	\item Email your solutions to \href{mailto:samf.training.assignments@gmail.com}{\texttt{samf.training.assignments@gmail.com}}.
	\item In the subject of your email, include your name and the level of the assignment (Beginner, Intermediate or Senior).
	\item Submit each question in a single separate PDF file (with multiple pages if necessary), with your name and the question number written on each page.
	\item If you take photographs of your work, use a document scanner such as Office Lens to convert to PDF.
	\item If you have multiple PDF files for a question, combine them using software such as PDFsam.
\end{itemize}

\end{document}