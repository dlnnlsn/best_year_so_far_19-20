\documentclass{article}

\usepackage{amsmath,amssymb}
\usepackage{hyperref}

\begin{document}

\begin{center}
\textbf{\Large Senior January Monthly Problem Set}
\\ \vspace{1em}
\textbf{\large Due: Friday, 17 January 2020}
\end{center}

\begin{enumerate}

\vspace{6pt}
\item[1.] %PAMO Shortlist 2019
On the board, we write the integers $1, 2, 3, \dots, 2019$. At each minute, we pick two numbers on the board $a$ and $b$, erase them, and write down the number $s(a + b)$ instead where $s(n)$ denotes the sum of the digits of the integer $n$. Let $N$ be the last number remaining on the board.
\begin{enumerate}
	\item Is it possible that $N = 19$?
	\item Is it possible that $N = 15$?
\end{enumerate}

\vspace{6pt}
\item[2.] %Baltic Way 2009 Problem 8
For which positive integers $n$ is it possible to divide the numbers $n, (n + 1), (n + 2), \dots, (n + 8)$ into two disjoint sets $A$ and $B$ such that the product of the numbers in $A$ is equal to the product of the numbers in $B$?

\vspace{6pt}
\item % Liam
Let $M$ be a positive integer, and let $S$ denote the set of finite sequences of positive integers, including the empty sequence of length zero, which we denote as $\mathbf{0}$. Also, for a sequence $\mathbf{x} = (x_1, x_2, \dotsc, x_{n-1}, x_n)$ let $\overline{\mathbf{x}}$ denote the reverse sequence $\overline{\mathbf{x}} = (x_n, x_{n-1}, \dotsc, x_2, x_1)$.

Define a function $d$ from finite sequences of positive integers to the integers as follows:
\begin{itemize}
\item $d(\mathbf{0}) = 0$.
\item If $\mathbf{x} = (x_1, x_2, \dotsc, x_n)$ is a sequence of positive length, let $m$ be the largest integer such that $x_1 +x_2 +\dotsb +x_m \leq M$, and let $\mathbf{x}'$ denote the rest of the sequence: $\mathbf{x}' = (x_{m+1}, \dotsc, x_n)$. Then $d(\mathbf{x}) = 1 +d(\mathbf{x}')$.
\end{itemize}
Show that $d(\mathbf{x}) = d(\overline{\mathbf{x}})$.


\vspace{6pt}
\item[4.]

\vspace{6pt}
\item[5.]

\vspace{6pt}
\item[6.]

\vspace{6pt}
\item[7.]

\vspace{6pt}
\item[8.]

\end{enumerate}

\vfill
\textbf{\Large Email submission guidelines}
\begin{itemize}
	\item Email your solutions to \href{mailto:samf.training.assignments@gmail.com}{\texttt{samf.training.assignments@gmail.com}}.
	\item In the subject of your email, include your name and the level of the assignment (Beginner, Intermediate or Senior).
	\item Submit each question in a single separate PDF file (with multiple pages if necessary), with your name and the question number written on each page.
	\item If you take photographs of your work, use a document scanner such as Office Lens to convert to PDF.
	\item If you have multiple PDF files for a question, combine them using software such as PDFsam.
\end{itemize}

\end{document}
