\documentclass{article}

\usepackage{amsmath,amssymb}
\usepackage{fullpage}
\usepackage{enumerate}
\usepackage{hyperref}
\usepackage{graphicx}


\begin{document}

\setlength{\tabcolsep}{0.015\textwidth}
\begin{center} \begin{tabular}{cccc}
	\includegraphics[width=0.16\textwidth]{SAMF_logo.jpg} &
	\includegraphics[width=0.35\textwidth]{SAICA_logo.jpg} &
	\includegraphics[width=0.18\textwidth]{Liberty_logo.jpg} &
	\includegraphics[width=0.18\textwidth]{SAMO2019.png}
\end{tabular} \end{center}


\bigskip \bigskip

\begin{center}
\textbf{\Large January Monthly Problem Set}
\\ \vspace{1em}
\textbf{\large Due: 24 January 2020}
\end{center}

\begin{enumerate}[1.]

\vspace{6pt}
\item


\vspace{6pt}
\item


\vspace{6pt}
\item % Liam
Let $M$ be a positive integer, and let $S$ denote the set of finite sequences of positive integers, including the empty sequence of length zero, which we denote as $\mathbf{0}$. Also, for a sequence $\mathbf{x} = (x_1, x_2, \dotsc, x_{n-1}, x_n)$ let $\overline{\mathbf{x}}$ denote the reverse sequence $\overline{\mathbf{x}} = (x_n, x_{n-1}, \dotsc, x_2, x_1)$.

Define a function $d$ from finite sequences of positive integers to the integers as follows:
\begin{itemize}
\item $d(\mathbf{0}) = 0$.
\item If $\mathbf{x} = (x_1, x_2, \dotsc, x_n)$ is a sequence of positive length, let $m$ be the largest integer such that $x_1 +x_2 +\dotsb +x_m \leq M$, and let $\mathbf{x}'$ denote the rest of the sequence: $\mathbf{x}' = (x_{m+1}, \dotsc, x_n)$. Then $d(\mathbf{x}) = 1 +d(\mathbf{x}')$.
\end{itemize}
Show that $d(\mathbf{x}) = d(\overline{\mathbf{x}})$.


\vspace{6pt}
\item


\vspace{6pt}
\item


\vspace{6pt}
\item


\vspace{6pt}
\item


\vspace{6pt}
\item


\end{enumerate}


\vfill
\textbf{\Large Email submission guidelines}
\begin{itemize}
	\item Email your solutions to \href{mailto:samf.training.assignments@gmail.com}{\texttt{samf.training.assignments@gmail.com}}.
	\item In the subject of your email, include your name and the level of the assignment (Beginner, Intermediate or Senior).
	\item Submit each question in a single separate PDF file (with multiple pages if necessary), with your name and the question number written on each page.
	\item If you take photographs of your work, use a document scanner such as Office Lens to convert to PDF.
	\item If you have multiple PDF files for a question, combine them using software such as PDFsam.
\end{itemize}

\end{document}